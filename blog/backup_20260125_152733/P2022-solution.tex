\documentclass{article}
\usepackage[UTF8]{ctex}
\usepackage{hyperref}

\author {AtomAlpaca}
\title  {「题解」 P2022 有趣的数}
\begin{document}
	\tableofcontents
	\par
	逝去的从容逝去,重温的依然重温,在沧桑的枝叶间,折取一朵明媚,簪进岁月肌里,让它疼痛又甜蜜,让它流去又流回。 ——汪曾祺《人间草木》
	\subsection {题目}
	\par
	\href {https://www.luogu.com.cn/problem/P2022} {link}
	\subsection {分析}
	\par
	
	这道题目要求我们做这样几件事:
	
	\begin{enumerate}
		\item 判断是否存在 $N$ 使得 $Q \left( N,K \right)=M$;
		\item 如果存在,求出 $N$ 的最小值;
		\item 如果不存在,输出 $0$.
	\end{enumerate}
	
	\par
	通过分析 \href {https://zh.wikipedia.org/zh-cn/字典序} {字典序} 的定义,我们知道:
	$$
		Q \left( I, K \right) \ge Q \left( N, K \right), I > N
	$$
	
	\par
	这是显然的, 因为随着 $I$ 的增大, $K$ 字典序前面的数字绝对不会跑到它后面去. 对于不是 $10^n$ 的形式的数字, 它前面的数字显然会可能增加. 这就意味着, $K$ 一定时, 随着 $M$ 的增大, $ Q \left( M, K \right) $ 单调递增.
	
	\par
	由于 $K$ 必须要出现在字典序列中, $M$ 的最小值为 $K$. 此时$ Q \left( M, K \right) $也最小. 因此我们知道:
	\begin{enumerate}
		\item 如果 $ Q \left( K, K \right) > M$ , 那么一定不存在 $N$ 使 $ Q \left( N, K \right)=M$;
		\item 如果 $ Q \left( K, K \right) = M$ , 那么所求 $ N = K $.
	\end{enumerate}

	\par
	对于形如 $ 10 ^ {n} $ 的 $K$, 显然 $ Q \left( K, K \right) = n + 1$. 但对于一般数字, 我们首先需要设法求 $ Q \left( K, K \right) $

	\par
	如果我们求出在 $k$ 前的数字的数量,加 $1$ 便是 $ Q \left( K, K \right)$. 我们从位数从小到大的角度讨论在其字典序前的数字.

	\par
	根据字典序的定义, 对于位数为 $l_n$ 的数字 $n$, 我们只需要与 $K$ 的前 $l_n$ 位进行比较. 我们预先处理出 $K$ 的前 $l$ 位 $K_l$, 此时不难发现当且仅当 $ n \le K_l $ 时, $n$ 在 $K$ 前, 或 $n$ 就是 $K$ 本身. 又考虑到任意数字开头不为零, 因此, 在所有位数为 $l$ 的数字中, 排在 $K$ 前或与 $K$ 相等的有 $ K_{l} - 10 ^ {l - 1} + 1 $ 个.

	\par
	因此我们得到:
	$$
		Q \left( K, K \right) = \sum_{ l = 1 } ^ { L }  {{K_l} - 10 ^ {l - 1} + 1}
	$$
	其中 $L$ 是 $K$ 的位数.
	
	\par
	注意这里其实算进来了一个等于 $K$ 的数, 但是因为最后要加 $1$, 所以二者可相互抵消.

	\par
	当 $ Q \left( K, K \right) < M $ 时, 我们还要求出 $N$ 令 $ Q \left( N, K \right) = M$. 考虑到对于任意 $L$ 位数 $n$, 若 $n > K$, $n$ 一定排在 $K$ 的字典序后, 因此我们直接讨论 $L + 1$位数.

	\par
	我们发现,所有小于 $K$ 的 $L$ 位数, 在后面加上一个数字后, 得到的数字的字典序一定还在 $K$ 的字典序之后,因此每增加一位,  $K$ 字典序前的数字增加的量就等于上一位数 $K$  字典序前的数字增加的量的 $10$ 倍. 由此我们得到, 将序列增加到 $L + i$ 位数时, $K$ 字典序前的数字会增加 $ K_{l} \times 10 ^ {i} - 10 ^ {l + i - 1} $ 个.

	\par
	令 $ P = M - Q \left( K, K \right) $, 我们不断枚举 $i$ ,让 $P$ 每次减 $ K_{l} \times 10 ^ {i} - 10 ^ {l + i - 1} $, 当第 $r$ 次减去后 $ P < 0 $ 时, 记录 $r - 1$ 及当时的 $P$, 则答案为
	$$
		10^{r - 1} + P - 1
	$$

	\par
	这是因为, 当枚举到 $r - 1$ 时, 我们距离目标 $M$ 还有 $P$ 个数字, 这时我们加上这 $P$ 个 $r$ 位数, 在减去与 $K$ 相等的那个数, 我们就能得到答案.

	\subsection{代码}
	\begin{verbatim}
#include <iostream>

using std::cin;
using std::cout;

long long k, m, min, now, len;
long long power[20],  nums[20];
int main()
{
	cin >> k >> m;
	power[0] = 1;

	for (int i = 1; i <= 20; ++i)
	{
		power[i] = power[i - 1] * 10;
	}

	for (int i = 0; i <= 20; ++i)
	{
		if (k == power[i] and m != i + 1)
		{
			cout << 0;
			return 0;
		}
	}

	std::string str;
	str = std::to_string(k);
	len = str.length();

	for (int i = 1; i < len; ++i)
	{
		nums[i] = k / power[(len - i)];
	}

	nums[0] = 0;
	nums[len] = k;

	for (int i = 1; i < 11; ++i)
	{
		if (k < power[i - 1])
		{
			break;
		}
		min += nums[i] - power[i - 1] + 1;
	}

	if (min > m)
	{
		cout << 0;
	}
	else if (min == m)
	{
		cout << k;
	}
	else
	{
		m -= min;
		long long ans = power[len];
		int i = 1;
		while (true)
		{
			long long tmp = k * power[i] - power[len + i - 1];
			if (m > tmp)
			{
				m -= tmp;
				ans *= 10;
				++i;
			}
			else
			{
				break;
			}
		}
		ans += m - 1;
		cout << ans;
	}
	return 0;
}
	\end{verbatim}
\end{document}
