\documentclass{article}
\usepackage[UTF8]{ctex}
\usepackage{hyperref}
\usepackage{graphicx}

\author {AtomAlpaca}
\title  {「思考」有关“精致笔记”}
\begin{document}
\tableofcontents

\subsection{前言}

似乎从我上网冲浪开始,对“精致笔记”的口诛笔伐就层出不穷,仿佛“把笔记记得好看”已经成为了没有“学习方法”的代名词。例如 \href{https://www.zhihu.com/question/320065759}{这个},\href{https://www.zhihu.com/question/33971405}{这个}。

回头看看自己的博客,似乎也充斥这这样的“精致笔记”。故再次写下此篇“精致笔记”,梳理我对笔记(及其衍生——博客)的看法。

\subsection{“给自己看的笔记”和“给别人看的笔记”}
我的笔记分为“自己看的”和“别人看的”。

前者或许连笔记都不算,更多的是以教科书上混乱的勾画、横飞的箭头和挤在空隙里的字迹之类的形式出现。这种“非精致笔记”多是在上课期间匆忙写下的——我得承认我的高中老师讲课风格可谓劲若奔星,想做“精致笔记”实在来不及,连这些“非精致笔记”都会打扰我的思路。这也是网上对“精致笔记”颇有微词的一大原因,如果你在课上就能写出精美如手帐一般的笔记,那你真的听课了吗。

这些知识碎片混乱得几乎只能归类为涂鸦,但我能看懂,也只有我自己看,这是属于我自己的“提示词”,我自己的“trigger”,其他人看了再云里雾里也不重要。

其余的笔记,多分布在活页的笔记本,和你正看的这个博客上。这些“精致笔记”就是写给别人看的,我写完后可能自己再也不会看,并且我不认为我这样做是错的。

正如\href{https://blog.atal.moe/post/type.html}{类型论“笔记”}前言中说的,我希望我的“精致笔记”可以作为“入门读物”或者“通俗教材”一类的东西。我认为“把这些东西给一个八岁小孩讲懂”是梳理知识的最佳方式,有人把这个叫“费曼学习法”或者“门徒效应”什么的。写这种笔记的时候你会想到许多学习的时候很难去想的东西,比如:

\begin{enumerate}
	\item 我们引入了一个概念或方法,但我们为什么要引入,它的动机是什么?
	\item 这样做是对的,但为什么是对的?
	\item 这样做是对的,但怎么能想到要这么做?
	\item 这里用了这样的处理方法,还有什么相关的问题也用了这种方法吗?
	\item 其他人学到这里还会提出什么疑问吗?
\end{enumerate}

这种对自己知识体系查缺补漏和建立知识之间联系的过程是重要的,将“非精致笔记”连缀成“精致笔记”的过程甚至重要过笔记本身。

\includegraphics[width=0.80\textwidth]{https://s2.loli.net/2025/01/22/MW1EqNmrz4QyevR.jpg}

例如这是我自己的一页“精致数学笔记”。在写这篇笔记的时候我让自己思考了:

\begin{enumerate}
	\item 我怎么得到的参数方程?
	\item 焦点在 $x$ 轴和 $y$ 轴的参数方程不一样,为什么这样设计?
\end{enumerate}

\subsection{你能超越教材吗}

与“做‘精致笔记’有价值吗”相对的,“精致笔记”本身的价值也倍受怀疑。我们能比教材写得更好吗?我们的“精致笔记”价值能超过教材复印件吗?

我认为答案是肯定的。一方面个人写出的笔记常带有个人思想的脉络。如果读者使用过苏联式的理工教材,可能会感受到这些教材常常把具体细节和知识动机隐藏得很好——这便是个人笔记优越的地方。另一方面,一些细致末节的问题可能没有人会讲,也许没有人想过,也许不屑于讲——你有想过为什么树状数组每次要加减 $lowbit$ 而不是其它东西吗?——这种事情往往只能在素未相识的某人的博客里得到答案。


\subsubsection{自由的知识}

这部分有点强行上价值了。

近年来信息学竞赛呈现这样一个趋势:对高级算法的考察减少,而对各种 Ad-hoc 题目考察增多,这导致一个问题:很多思维上的 trick 没见过就会被初见杀,见过题目难度就会大幅下降。这使得强校垄断知识成为可能——毕竟出题人都是这些强校走出来的,只要它们把新的 trick 按在手里不公开,只在校内传播,就能给自己带来极大的优势。据说类似的现象在其余四个学科竞赛中早有体现。

而公开的“精致笔记”,作为一种“教材平替”,在此时竟然有可能成为一个学阀垄断高墙上的窗口。

\subsection{自我感动,不好吗}

对“精致笔记”批判的另一个常见角度是“你做这些笔记就是为了感动自己”。

感动自己不好吗。让自己开心不好吗。学点东西非得像苦行僧一样吗。如果做笔记能让自己对这个领域更感兴趣为什么不呢。

不是每个人都是圣人,不是每个人都能在没有正反馈的情况下一直自律,我最开始建立博客就是为了满足自己的虚荣心。“哇快看我有一个博客,我太酷了”,“哇我写了这么多博文快夸夸我”,“我在洛谷上写了题解我的社区贡献要涨了”。但即使建立博客的动机不纯,但并不妨碍这种虚荣心能给我梳理知识的动力,这些博客也不可否认地帮助了我加深对某些知识的理解。我就是在自我感动,不行吗。

\subsection{后记}

不知道为什么这篇写得像给自己辩护“我是有学习方法的不是抄书做手帐的无脑人群”,又像在说“你看我的学习方法多好我多伟大”,总觉得怪怪的。

\end{document}