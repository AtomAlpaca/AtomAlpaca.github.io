\documentclass{article}
\usepackage[UTF8]{ctex}
\usepackage{hyperref}
\usepackage{graphicx}

\author {AtomAlpaca}
\title  {「题解」UVA754 Treasure Hunt}
\begin{document}
    \tableofcontents
    \subsection {题目}
    \href {https://www.luogu.com.cn/problem/UVA754} {link}
    \subsection {分析}
		首先考虑证明这样一个事实:在终点相同的时候,仅为一条直线的路线一定不劣于多条直线拼接成的路线。

\par
我们将一条多条直线拼接成的路线视作多个不同方向的凸包的拼接。将这些包的首尾点相连,如图:
\par
\includegraphics[width=0.80\textwidth]{https://cdn.luogu.com.cn/upload/image_hosting/kks2v25n.png}

\par
然后我们发现一个显然的结论:和凸包相交但是和直线不相交的直线是存在的;而和直线相交而和凸包不相交的直线不存在。

\par
因此我们考虑反复用直线替换凸包直至路线变为一条直线,这样的方案一定是不劣于原方案的。

\par
\includegraphics[width=0.80\textwidth]{https://cdn.luogu.com.cn/upload/image_hosting/9hw3k6vz.png}

\par
有了这个结论,我们就可以考虑枚举边界上的点,和目标节点连接得到路线。
\par
考虑到答案变化当且仅当这条直线在“旋转”的时候“扫过”了一个墙壁的某个端点,边界上的端点之间的所有点的答案是相同的,因此我们只需要枚举墙壁的端点即可。

	\subsection{代码}
	\begin{verbatim}
#include <bits/stdc++.h>

using std::cin;
using std::cout;

typedef double db;
const db eps = 1e-8;
const int MAX = 35;
int cmp(db x) { if (fabs(x) <= eps) { return 0; } if (x > eps) { return 1; } return -1; }
struct P
{
  db x, y;
} p, p1[MAX], p2[MAX];
const int INF = 0x3f3f3f3f;
int ans, T, n;

typedef P V;
inline P operator + (P a, P b) { return {a.x + b.x, a.y + b.y}; }
inline P operator - (P a, P b) { return {a.x - b.x, a.y - b.y}; }
inline db dot(P a, P b) { return a.x * b.x + a.y * b.y; }
inline db cro(P a, P b) { return a.x * b.y - a.y * b.x; }
bool cross(P a, P b, P c, P d)
{
  db c1 = cro(b - a, c - a), c2 = cro(b - a, d - a);
  db d1 = cro(d - c, a - c), d2 = cro(d - c, b - c);
  return (cmp(c1) * cmp(c2) < 0) and (cmp(d1) * cmp(d2) < 0);
}

void solve()
{
  ans = INF;
  cin >> n; 
  for (int i = 1; i <= n; ++i) { cin >> p1[i].x >> p1[i].y >> p2[i].x >> p2[i].y; }
  cin >> p.x >> p.y;
  for (int i = 1; i <= n; ++i)
  {
    int ans1 = 0, ans2 = 0;
    for (int j = 1; j <= n; ++j)
	{
	  if (cross(p, p1[i], p1[j], p2[j])) { ++ans1; }
	  if (cross(p, p2[i], p1[j], p2[j])) { ++ans2; }
	}
	ans = std::min(ans, ans1);
	ans = std::min(ans, ans2);
  }
  if (!n) { cout << "Number of doors = 1\n"; }
  else { cout << "Number of doors = " << ans + 1 << '\n'; }
  if (T) { cout << '\n'; }
}

int main()
{
  cin >> T;
  while (T--) { solve(); }
  return 0;
}
	\end{verbatim}
\end{document}
