\documentclass{article}
\usepackage[UTF8]{ctex}
\usepackage{hyperref}
\usepackage{fancyvrb}
\usepackage{amsmath}
\hypersetup {
	colorlinks = true,
}

\newcommand{\[}{\begin{align*}}
	\newcommand{\]}{\end{align*}}
\renewcommand{\par}[1]{\paragraph{#1}}

\author {AtomAlpaca}
\title {「信创」如何给在 Windows XP 上玩 ncnn}

\begin{document}
	\tableofcontents
	\subsection{前前言}
本文所写的与最终合并进 \texttt{ncnn} 仓库的代码有所出入,切后续可能进行更改,本文内容仅供参考,以实际代码为准。
	\subsection{前言}
笔者很菜,第一次做类似的工作,求轻点喷。

首先几个主要的难点:

\begin{enumerate}
	\item  32 位的 Windows XP 很常见,我们不得不考虑编译成 32 位程序
	\item  现代的某些 Windows 系统函数在 Windows XP 时代不存在
\end{enumerate}

也有好消息:Windows XP 从系统内核层面上不支持 AVX 指令集,否则不知道又要多出多少个坑。

\subsection{使用 mingw 构建}

首先我们需要一个能编译至 32 位的版本。我们来到 mingw-w64 的 sourceforge,一路找到 \texttt{oolchains targetting Win32/Personal Builds/mingw-builds/8.1.0/threads-posix/dwarf/i686-8.1.0-release-posix-dwarf-rt_v6-rev0.7z}

这里的 \texttt{threads} 必须位 posix 而不是 win32,不然会无法使用 \texttt{std::thread}。

为了方便找到特定的工具链我们不妨设置一个环境变量 \texttt{MINGW32_ROOT_PATH} 到这个工具链的根文件夹,然后在 \texttt{toolchains} 下创建一个 cmake 文件


\begin{verbatim}
set(CMAKE_SYSTEM_NAME Windows)
set(CMAKE_SYSTEM_PROCESSOR x86_64)

if(DEFINED ENV{MINGW32_ROOT_PATH})
file(TO_CMAKE_PATH $ENV{MINGW32_ROOT_PATH} MINGW32_ROOT_PATH)
else()
message(FATAL_ERROR "MINGW32_ROOT_PATH env must be defined")
endif()

if(DEFINED ENV{OPENCV_MINGW_DIR})
file(TO_CMAKE_PATH $ENV{OPENCV_MINGW_DIR} OpenCV_DIR)
endif()

set(MINGW32_ROOT_PATH ${MINGW32_ROOT_PATH} CACHE STRING "root path to mingw toolchain")

set(CMAKE_C_COMPILER "${MINGW32_ROOT_PATH}/bin/i686-w64-mingw32-gcc.exe")
set(CMAKE_CXX_COMPILER "${MINGW32_ROOT_PATH}/bin/i686-w64-mingw32-g++.exe")

set(CMAKE_FIND_ROOT_PATH "${MINGW32_ROOT_PATH}/i686-w64-mingw32")

if(NOT CMAKE_FIND_ROOT_PATH_MODE_PROGRAM)
set(CMAKE_FIND_ROOT_PATH_MODE_PROGRAM NEVER)
endif()
if(NOT CMAKE_FIND_ROOT_PATH_MODE_LIBRARY)
set(CMAKE_FIND_ROOT_PATH_MODE_LIBRARY ONLY)
endif()
if(NOT CMAKE_FIND_ROOT_PATH_MODE_INCLUDE)
set(CMAKE_FIND_ROOT_PATH_MODE_INCLUDE ONLY)
endif()
if(NOT CMAKE_FIND_ROOT_PATH_MODE_PACKAGE)
set(CMAKE_FIND_ROOT_PATH_MODE_PACKAGE ONLY)
endif()
\end{verbatim}

对于 Windows XP,我们需要指定几个比较关键的编译参数:
\begin{enumerate}
\item  \texttt{-D_WIN32_WINNT=0x0501},\texttt{_WIN32_WINNT} 指定了所使用的最晚 Windows SDK 版本,\texttt{0x0501} 是 Windows XP 的代码,这个参数能禁用所有 Windows XP 不支持的 Windows 提供的函数。\href{https://learn.microsoft.com/zh-cn/cpp/porting/modifying-winver-and-win32-winnt?view=msvc-170}{Content}。
\item  \texttt{-march=i686} 指定编译的目标架构为 \texttt{i686}(即 32  位平台)。
\item  \texttt{-static} 尽量静态编译减少对第三方库的依赖 
\end{enumerate}


其次链接的时候要加上 \texttt{-static -fopenmp},否则会报找不到 \texttt{libgomp-1.dll}。

其次是代码中几处需要调整的地方:
platform.h.in 中的 \texttt{Mutex} 和 \texttt{ConditionVariable} 分别封装了 \texttt{SRWLOCK} 和 \texttt{CONDITION_VARIABLE},而这两个东西到 Windows 7 才出现。于是我们需要替代一下。直接替换成空实现不太道德,我们使用 \texttt{CRITICAL_SECTION} 和事件简单实现一下:


\begin{verbatim}
class NCNN_EXPORT Mutex
{
	public:
	Mutex() { InitializeCriticalSection(&cs); }
	~Mutex() { DeleteCriticalSection(&cs); }
	void lock() { EnterCriticalSection(&cs); }
	void unlock() { LeaveCriticalSection(&cs); }
	private:
	friend class ConditionVariable;
	CRITICAL_SECTION cs;
};

class NCNN_EXPORT ConditionVariable
{
	public:
	ConditionVariable() { event = CreateEvent(0, FALSE, FALSE, 0); }
	~ConditionVariable() { CloseHandle(event); }
	void wait(Mutex& mutex)
	{
		mutex.unlock();
		WaitForSingleObject(event, INFINITE);
		mutex.lock();
	}
	void broadcast() { SetEvent(event); }
	void signal() { SetEvent(event); }
	private:
	HANDLE event;
};
\end{verbatim}

再改一下宏的判断就行了,应该没什么问题。

\texttt{test_cpu.cpp} 里,原本写得是 \texttt{#if _WIN32_WINNT >= _WIN32_WINNT_WIN7`},但是似乎没有用到 Windows XP 不支持的操作,直接改掉就好了。
3. 如果你还想用 vlukan 的话,\texttt{glslang} 里用到了 \texttt{_itoa_s} 和 \texttt{_vsnprintf_s},这些是 Windows XP 下没有的、后增加的安全版本的函数。我们注意到这里是否使用安全函数是一个宏 \texttt{MINGW_HAS_SECURE_API} 控制的,一般来讲这个宏又会被 \texttt{_CRT_SECURE_NO_DEPRECATE} 控制,但是不知道为什么我们这里的 \texttt{MINGW_HAS_SECURE_API} 被写死成了 1,我们可以稍微修改一下这里的判断,改成类似 \texttt{#if !(defined(_WIN32_WINNT) && _WIN32_WINNT <= 0x0501) && (defined(MINGW_HAS_SECURE_API) && MINGW_HAS_SECURE_API)}。感觉这个很 dirty,但是他工作了。

似乎也没改什么东西。

至此我们可以开始着手编译了。注意虽然 Windows XP 是不支持 \texttt{avx} 的但是我们选择的编译器支持,于是我们要手动把 \texttt{-DNCNN_AVX} 和 \texttt{-DNCNN_AVX} 设置为 \texttt{OFF}。

\begin{verbatim}
cmake -DCMAKE_TOOLCHAIN_FILE="../toolchains/windows-xp-mingw.toolchain.cmake" -DCMAKE_BUILD_TYPE=debug -DNCNN_RUNTIME_CPU=OFF -DNCNN_BUILD_BENCHMARK=ON -DNCNN_BUILD_EXAMPLES=OFF -DNCNN_BUILD_TESTS=OFF -DNCNN_SIMPLEOCV=ON -DNCNN_AVX2=OFF -DNCNN_AVX=OFF -DNCNN_VULKAN=ON .. -G "MinGW Makefiles"
cmake --build . -j 4
\end{verbatim}

这里直接使用了 \texttt{-DNCNN_SIMPLEOCV},就不用在 Windows XP 上配置 \texttt{OpenCV} 了,我们之后也会使用这个编译选项。

\subsubsection{我就想在 Windows XP 上配 OpenCV}

别吧。

笔者折腾了半天没折腾出来,决定放一放。折腾出来了再补。

\subsection{用 Clang 构建}

其实不建议这么做。Clang 发行时默认不包含 C 运行时库,也不包含构建 Windows XP 应用程序所需的头文件和库,我们需要借用 Mingw 的 32 位库进行构建。因此其实直接用 mingw 构建更为方便。

官方明确了最后一个支持 Windows XP 的版本是 3.7.0,详见 \href{(https://releases.llvm.org/3.7.0/tools/clang/docs/ReleaseNotes.html#last-release-which-will-run-on-windows-xp-and-windows-vista}{Release Note}。我们下载这个版本,安装并且添加环境变量。
(后记:后面证实了其实更晚一些的版本仍然能正常工作)

和 mingw 不同的几个点:
\begin{enumerate}
\item \texttt{target} 选项改成 \texttt{--target=i686-pc-windows-gnu}
\item 添加 \texttt{--sysroot=\$\{MINGW32_ROOT_PATH\}} 强制使用 Mingw32 的库
\item 由于不支持 \texttt{__float128`,需要加上 `-D__STRICT_ANSI__`}
\item 由于这个版本下 openmp 实在无法正常工作,不得不使用 SAMPLEOMP,将 \texttt{-fopenmp} 移除。
\item 这个版本的 clang 不支持 \texttt{const T t;}这种写法,必须提供一个构造器如 \texttt{const T t = {};}。这部分问题出在 \texttt{src/layer/binaryop.cpp} 和 \texttt{src/layer/x86/binaryop_x86.cpp} 两个文件中,把这里的 \texttt{const Op op;} 改成 \texttt{const Op op = {};}。虽然这样做会多出两条指令,但是开启优化之后能够优化掉。你可以试试 \href{https://godbolt.org/z/5796bWW9G}{上手玩玩}。
\end{enumerate}


之后就可以正常编译了:

```bash
\begin{verbatim}
cmake -DCMAKE_TOOLCHAIN_FILE="../toolchains/windows-xp-clang.toolchain.cmake" -DCMAKE_BUILD_TYPE=debug -DNCNN_RUNTIME_CPU=OFF -DNCNN_BUILD_BENCHMARK=ON -DNCNN_BUILD_EXAMPLES=ON -DNCNN_BUILD_TESTS=ON -DNCNN_SIMPLEOCV=ON -DNCNN_SIMPLEOMP=ON -DNCNN_SSE2=OFF -DNCNN_AVX2=OFF -DNCNN_AVX=OFF .. -G "MinGW Makefiles"
cmake --build . -j 4
\end{verbatim}

编译出的 example 和 benchmark 应当都能在 Windows XP 上正常运行了。

\subsection{用 VS2017 构建}
官方给出的最后一个支持 Windows 开发的版本时 VS2017,\href{https://learn.microsoft.com/en-us/cpp/porting/features-deprecated-in-visual-studio?view=msvc-170}{详情点击即看}。此外我们还需要\href{https://stackoverflow.com/questions/49516896/how-to-install-build-tools-for-v141-xp-for-vc-2017}{额外安装 \texttt{v141_xp} 工具集}才能正常构建。

几个要注意的点:

\begin{enumerate}
\item MSVC 传入参数的方式不同,应该使用 \texttt{/D_WIN32_WINNT=0x0501} 这种形式;
\item 一个可有可无的点:MSVC 会觉得你没有正常处理异常,于是会报很多 warning。我们用 \texttt{/EHsc} 钦定异常只在 throw 语句或函数调用处发生他就安静了。
\item 需要静态链接运行时库,可以直接传一个 \texttt{/MT}(或者 debug 版本 \texttt{MTd}),或者 NCNN 也提供了 \texttt{DNCNN_BUILD_WITH_STATIC_CRT} cmake 选项。
\item 要用 \texttt{-A WIN32} 指定生成平台
\end{enumerate}


然后就能正常编译了:

\begin{verbatim}
mkdir build
cd build
cmake -A WIN32 -G "Visual Studio 15 2017" -T v141_xp -DNCNN_SIMPLEOCV=ON -DNCNN_OPENMP=OFF -DNCNN_AVX2=OFF -DNCNN_AVX=OFF -DNCNN_BUILD_WITH_STATIC_CRT=ON -DCMAKE_TOOLCHAIN_FILE="../toolchains/windows-xp-msvc.toolchain.cmake" ..
cmake --build . --config Release -j 2
cmake --build . --config Release --target install
\end{verbatim}

速度比 mingw 快多了。
\end{document}
