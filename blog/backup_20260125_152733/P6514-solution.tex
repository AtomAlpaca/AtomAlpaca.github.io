\documentclass{article}
\usepackage[UTF8]{ctex}
\usepackage{hyperref}
\usepackage{graphicx}

\author {AtomAlpaca}
\title  {「题解」P6514 [QkOI#R1] Quark and Strings} 
\begin{document}
\subsection{题意}

维护一个字符串序列 $\{S_n\}$,有两种操作共 $q$ 次,设当前为第 $i$ 次操作。

`1 l r`,表示在所有编号在 $[l,r]$ 内的字符串末尾添加字符 $i$。

`2 l r`,表示询问所有编号在 $[l,r]$ 内的字符串的最长公共子序列长度。

$1\le n,q\le 10^5$

\subsection{题解}
发现每次往后面加的都是一个独一无二的字符,我们其实要求的就是 $[l, r]$ 内都含有的字符种类。

考虑用一个 $(t, l, r)$ 三元组来描述一个操作,分别表示操作的时间、左端点、右端点,那么对于一个询问操作 $x$,我们其实就是要求 $t_y < t_x, l_y \le l_x, r_y \ge r_x$ 的修改操作 $y$ 的数量,这就是一个三维偏序的形式,我们用 CDQ 分治做一下就好了。复杂度 $O(n \log^2{n})$。代码相当简洁。

\subsection{代码}

\begin{verbatim}
#include <bits/stdc++.h>

const int MAX = 1e5 + 5;
int n, q, op, l, r, t;
int ans[MAX];
struct N { int t, l, r, x; } a[MAX];

bool cmpl(N a, N b) { return a.l == b.l ? a.r > b.r  : a.l < b.l; }
bool cmpt(N a, N b) { return a.t == b.t ? cmpl(a, b) : a.t < b.t; }

struct BIT
{
	int t[MAX];
	int lbt(int x) { return x & -x; }
	void add(int x) { if (!x) { return ; } while (x <= n) { ++t[x]; x += lbt(x); } }
	int qry(int x) { int res = 0; while (x) { res += t[x]; x -= lbt(x); } return res; }
	void clear() { for (int i = 1; i <= n; ++i) { t[i] = 0; } }
} st;

void solve(int l, int r)
{
	if (l == r) { return ; }
	int k = l + ((r - l) >> 1); solve(l, k); solve(k + 1, r);
	std::sort(a + l, a + k + 1, cmpl); std::sort(a + k + 1, a + r + 1, cmpl);
	int L = l, R = k + 1; st.clear();
	for (; R <= r; ++R)
	{
		while (L <= k and a[L].l <= a[R].l) { if (!a[L].x) { st.add(a[L].r); } ++L; }
		if (a[R].x) { ans[a[R].x] += st.qry(n) - st.qry(a[R].r - 1); }
	}
}

int main()
{
	scanf("%d%d", &n, &q);
	for (int i = 1; i <= q; ++i)
	{
		scanf("%d%d%d", &op, &l, &r);
		a[i] = {i, l, r, (op == 2 ? ++t : 0)};
	}
	solve(1, q);
	for (int i = 1; i <= t; ++i) { printf("%d\n", ans[i]); }
}
\end{verbatim}
\end{document}