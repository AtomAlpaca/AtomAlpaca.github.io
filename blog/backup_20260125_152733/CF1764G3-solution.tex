\documentclass{article}
\usepackage[UTF8]{ctex}
\usepackage{hyperref}
\usepackage{fancyvrb}
\usepackage{amsmath}
\hypersetup {
	colorlinks = true,
	linkcolor  = black,
}



\author {AtomAlpaca}
\title {「题解」CF1764G3 Doremy's Perfect DS Class}

\begin{document}

\subsection{题目}

交互题。每次产生一个 \([1,n]\) 的排列 \(p\)。

每次询问 \(l,r,k\),得到
\(\left\lfloor\dfrac{p_l}k\right\rfloor,\left\lfloor\dfrac{p_{l+1}}k\right\rfloor,\cdots,\left\lfloor\dfrac{p_r}k\right\rfloor\)
中数的种类数。

要求在 \(30\)(easy)/\(24\)(medium)/\(20\)(hard) 次询问内找到 \(1\)
的位置。\(n \le 1024\)。

很厉害的一道题目!然而自己只想到 medium 的做法。但是很喜欢!

\subsection{题解}

\subsubsection{easy version}

看到 \(1024\) 和 \(30\),感觉像是 \(3\log n\)。于是猜是二分。

然后发现 \(k=2\) 的情况相当于把每个 \(2x, 2x+1\)
配对,最后剩下无法配对的是 \(1\) 和 \(n\)(\(n\) 为偶数时)。考虑把
\(n\) 是奇数和偶数的情况分开解决。

当 \(n\) 为奇数时,我们考虑先把 \(1\)
删去,于是把这个序列分成两个区间时,两个区间内的、无法配对的数的数量是相等的。然后我们把
\(1\) 扔到某一边,这边的无法配对的数的数量会增加,我们就能判定 \(1\)
在哪一边了。

考虑怎么求无法配对的数的数量。设一次询问 \(l, r, k\) 的答案为
\(q(l, r, k)\),那么 \((r - l + 1) - q(l, r, k)\)
显然就是配对的对数,因此未配对的对数就是
\((r - l + 1) - 2\times ((r - l + 1) - q(l, r, k))\),也就是
\(2 q(l, r, k) - (r - l + 1)\)。

于是我们可以通过两次操作把目标范围缩小一半,并且操作次数是十分优秀的
\(O(2 \log n)\)。

然后考虑 \(n\) 为偶数怎么做。发现只是在奇数的情况下多了一个 \(n\)
的干扰。然后我们发现一个区间 \([l, r]\) 含有 \(n\) 当且仅当
\(q(l, r, k)=2\),于是我们可以提前把 \(n\)
的位置二分出来,然后套用奇数的方法,加下特判就好了。操作次数是并不十分优秀的
\(O(3 \log n)\)。但是能够通过 easy version。

\subsubsection{medium version}

考虑优化 easy version 的做法。

我们发现每次把排列划分成两段时,如果 \(1, n\)
在同侧,我们其实是不用处理的,直接往未配对较多的一边跳就行;在异侧时,我们得到的
\(1\) 的新范围内一定不会再有
\(n\)。于是我们发现异侧的情况最多仅有一次,而且我们只需要知道 \(n\)
在哪一边即可。于是我们可以把操作次数优化到 \(O(2 \log n + 1)\)。足以通过
medium version。

\subsubsection{hard version}

我们发现 \(O(2 \log n + 1)\) 最大只有 \(21\),恰好无法通过 hard
version。考虑继续优化。

考虑偶数情况一定会经过 \(r = l + 1\)
的区间,这个区间是较好处理的。分为两种情况。

\begin{enumerate}
\def\labelenumi{\arabic{enumi}.}
\item
  还未出现 \(1, n\) 异侧,即 \(l, r\) 对应的分别是 \(1, n\);
\item
  出现了 \(1, n\) 异侧,即 \(l, r\) 其中一个是 \(1\),另一个不一定。
\end{enumerate}

第一种情况是好处理的,通过询问 \(q(1, l, n)\) 或 \(q(r, n, n)\)
判断其中一个是否为 \(n\) 即可。第二种情况则稍难一些。

由于我们现在的区间为 \([l, r]\),则我们之前一定询问过
\(q(1, l - 1, 2), q(1, r, 2), q(l, n, n), q(r + 1, n, n)\)
这些区间。我们可以利用这些信息。

\begin{itemize}
\item
  假如 \(q(1, r, 2) = q(1, l - 1, 2) + 1\) 说明加上 \(l, r\) 之后,和
  \([1, l-1]\) 中的区间匹配上了一个数,于是判断 \(q(1, l, 2)\) 是否等于
  \(q(1, l - 1, 2)\) 便能得到 \(l\) 这个位置是和前面匹配上的数还是
  \(1\);
\item
  否则,一定有
  \(q(l, n, n) = q(r + 1, n, n) + 1\),这种情况的处理方式和上面是对称的。
\end{itemize}

于是我们在 \(r = l + 1\) 的区间上优化掉了一次操作,操作次数变为
\(O(2 \log n)\),可以通过 hard version。

\subsection{代码}\label{ux4ee3ux7801}

\begin{verbatim}
#include <bits/stdc++.h>

bool flg, lft;
int n;
std::map < std::pair<int, int>, int > mp;

int qry(int l, int r, int k)
{
  if (l == r) { return 1; }
  if (l > r) { return 0; }
  if (k == 2 and mp.find({l, r}) != mp.end()) { return mp[{l, r}]; }
  int res = 0;
  printf("? %d %d %d\n", l, r, k); fflush(stdout);
  scanf("%d", &res);
  if (k == 2) { mp[{l, r}] = res; }
  return res;
}

int solve1(int l, int r)
{
  if (l == r) { return l; }
  int k = l + ((r - l) >> 1), x = 2 * qry(1, k, 2) - k, y = 2 * qry(k + 1, n, 2) - (n - k);
  if (x > y) { return solve1(l, k); } else { return solve1(k + 1, r); }
}

int solve0(int l, int r)
{
  if (l == r) { return l; }
  if (r - l == 1)
  {
    if (!flg)
    { 
      if (l > 1) { return ((qry(1, l, n) == 2) ? r : l); }
      else { return qry(r, n, n) == 2 ? l : r; };
    }
    if (qry(1, r, 2) == qry(1, l - 1, 2) + 1)
    {
      if (qry(1, l, 2) == qry(1, l - 1, 2)) { return r; } else { return l; }
    }
    else
    {
      if (qry(r, n, 2) == qry(r + 1, n, 2)) { return l; } else { return r; }
    }
  }
  int k = l + ((r - l) >> 1); int x = 2 * qry(1, k, 2) - k, y = 2 * qry(k + 1, n, 2) - (n - k);
  if (x == y)
  {
    if (!flg)
    {
      if (qry(1, k, n) == 2) { flg = lft = true; } else { flg = true; }
    }
    if (lft) { solve0(k + 1, r); } else { return solve0(l, k); }
  }
  if (x > y) { return solve0(l, k); } else { return solve0(k + 1, r); }
}

int main()
{
  scanf("%d", &n);
  if (n & 1) { printf("! %d\n", solve1(1, n)); } else { printf("! %d\n", solve0(1, n)); }
}
\end{verbatim}

\end{document}