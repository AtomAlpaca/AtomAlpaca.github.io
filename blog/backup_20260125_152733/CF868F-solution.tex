\documentclass{article}
\usepackage[UTF8]{ctex}
\usepackage{hyperref}
\usepackage{graphicx}

\author {AtomAlpaca}
\title  {「题解」CF868F Yet Another Minimization Problem} 
\begin{document}
	\subsection{题意}
	
	给定一个序列 $a$,要把它分成恰好 $k$ 个子段。每个代价的费用是其中相同元素的对数。求所有子段的代价之和的最小值。
	
	\href{https://codeforces.com/problemset/problem/868/F}{link}
	
	\subsection{题解}
	
	首先有一个暴力的 dp:
	
	$$
	f_{i, j} = \min_{k = 2}^{i}{f_{k - 1, j - 1} + v(k, i)}
	$$
	
	其中 $v(l, r)$ 是 $[l, r]$ 这一子段的代价。
	
	这个形式让人不禁四边形不等式了起来。我们考虑证明 $v$ 满足四边形不等式。
	
	我们不妨设 $w(l, r, s, t)$ 是 $[l, r]$ 中的元素和 $[s, t]$ 中元素相等的对数。那么对于 $a \le b \le c \le d$
	
	$$
	\begin{aligned}
		& v(a, c) + v(b, d) - (v(a, d) + v(b, d))\\
		=& 2v(b, c) + v(a, b) + v(c, d) + w(a, b, b, c) + w(b, c, c, d) \\
		-& 2v(b, c) - v(a, b) - v(c, d) - w(a, b, b, c) - w(b, c, c, d) - w(a, b, c, d) \\
		=& -w(a, b, c, d) \\
		\le& 0
	\end{aligned}
	$$
	
	因此得到 $v$ 满足四边形不等式。于是我们可以利用决策单调性来做这个问题。
	
	现在问题在于我们很难 $O(1)$ 地求解 $v$。那么我们考虑用类似莫队的思路,维护一个桶每次移动两个端点。然后我们发现对于每个分治区间,我们指针移动的次数是区间长度的一半,而分治的深度一共是 $\log$ 层,因此算下来我们的指针只会移动 $O(nk \log n)$ 次。
	
	至此我们可以在 $(nk \log n)$ 的时间复杂度解决这个问题。可以把 dp 的第二维用滚动数组优化掉,不过这题并不卡空间。
	
	\begin{verbatim}
#include <bits/stdc++.h>

typedef long long ll;
const int MAX = 1e5 + 5;
const ll INF = 1e16;
int n, k, L = 1, R = 0; ll res;
ll f[MAX][2]; int a[MAX], b[MAX];

ll w(int l, int r)
{
	while (R < r) { ++R; res += b[a[R]]++; }
	while (L > l) { --L; res += b[a[L]]++; }
	while (R > r) { res -= --b[a[R]]; --R; }
	while (L < l) { res -= --b[a[L]]; ++L; }
	return res;
}

void solve(int l, int r, int s, int t, int x)
{
	if (l > r) { return ; }
	int k = l + ((r - l) >> 1), p = s; ll mn = INF;
	for (int i = s; i <= std::min(k, t); ++i)
	{
		if (f[i - 1][x ^ 1] + w(i, k) < mn) { p = i; mn = f[i - 1][x ^ 1] + w(i, k); }
	}
	f[k][x] = mn;
	solve(l, k - 1, s, p, x); solve(k + 1, r, p, t, x);
}

int main()
{
	memset(f, 0x7f, sizeof(f)); f[0][0] = 0;
	scanf("%d%d", &n, &k);
	for (int i = 1; i <= n; ++i) { scanf("%d", &a[i]); }
	for (int i = 1; i <= k; ++i) { solve(1, n, 1, n, i & 1); }
	printf("%lld", f[n][k & 1]);
}
\end{verbatim}
\end{document}
