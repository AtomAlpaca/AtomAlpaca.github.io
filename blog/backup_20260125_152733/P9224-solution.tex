\documentclass{article}
\usepackage[UTF8]{ctex}
\usepackage{hyperref}
\usepackage{fancyvrb}
\usepackage{amsmath}
\hypersetup {
	colorlinks = true,
	linkcolor  = black,
}

\newcommand{\[}{\begin{align*}}
	\newcommand{\]}{\end{align*}}
\renewcommand{\par}[1]{\paragraph{#1}}


\author {AtomAlpaca}
\title {「题解」P9224 k 叉堆}

\begin{document}
	
	{
		\setcounter{tocdepth}{3}
		\tableofcontents
	}
	\subsection{题意}
	
	\href{https://www.luogu.com.cn/problem/P9224}{link}
	
	\subsection{题解}
	
	提供一个不需要根号分治的做法。
	
	首先考虑 \(k\) 叉堆的情况下,每个点的儿子的区间。考虑这样构造 \(k\) 叉堆:维护一个没被成为儿子的节点队列,从小到大考虑每个点,将队列中前 \(k\) 个树拎出来成为它的儿子。
	
	那么对于节点 \(n\),它前面有 \((n - 1)\times k\) 个节点被分配了,而 \(1\) 节点不能被分配为其它节点的儿子,所以节点 \(n\) 的儿子区间是 \([k(x - 1) + 2, kx + 1]\) 这一段。于是我们考虑对每一维信息维护一个 st 表,从小到大枚举叉数,检查每个节点的儿子区间内是否有比自己大的就可以了。
	
	复杂度是 \(O(k(\sum_{i=1}^{n-1}{\frac{n}{i}})) = O(knH_n)\),其中 \(H\) 是调和数。考虑到调和数增长很慢,几乎是 \(\log\) 级别的,这个做法应该是比出题人题解更优的。
	
	\subsection{代码}
	
	\begin{verbatim}
#include <bits/stdc++.h>
		
const int MAX = 2e5 + 5;
		
int n, k;
int a[MAX], st[21][MAX];
std::bitset <MAX> ans;
		
int mx(int l, int r)
{
	r = std::min(r, n); int lg = log2(r - l + 1);
	return std::max(st[lg][l], st[lg][r - (1 << lg) + 1]);
}
		
void solve()
{
	for (int i = 1; i <= n; ++i) { scanf("%d", &a[i]); st[0][i] = a[i]; }
	for (int i = 1; (1 << i) <= n; ++i)
	{
		for (int j = 1; j + (1 << i) - 1 <= n; ++j)
		{
			st[i][j] = std::max(st[i - 1][j], st[i - 1][j + (1 << (i - 1))]);
		}
	}
	for (int k = 1; k < n; ++k)
	{
		if (!ans[k]) { continue; }
		for (int i = 1; k * (i - 1) + 2 <= n; ++i)
		{
			if (mx(k * (i - 1) + 2, k * i + 1) > a[i]) { ans[k] = false; break; }
		}
	}
}

int main()
{
	scanf("%d%d", &n, &k);
	for (int i = 1; i <  n; ++i) { ans[i] = true; }
	for (int i = 1; i <= k; ++i) { solve(); }
	printf("%ld\n", ans.count());
	for (int i = 1; i <  n; ++i) { if (ans[i]) { printf("%d ", i); } }
}
	\end{verbatim}
	
\end{document}
