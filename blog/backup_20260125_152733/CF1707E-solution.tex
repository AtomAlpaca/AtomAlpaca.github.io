\documentclass{article}
\usepackage[UTF8]{ctex}
\usepackage{hyperref}
\usepackage{graphicx}

\author {AtomAlpaca}
\title  {「题解」CF1707E Replace}
\begin{document}
	\tableofcontents
錦瑟無端五十絃,一絃一柱思華年。

2023 年北京集训好题分享讲了这题。

\subsection{题意}

给定一个长为 $n$ 的序列 $a_n, \forall i \in [1, n], 1 \le a_i \le n$。

定义 $f(l,r)=(\min_{i=l}^{r}{a_i}, \max_{i=l}^{r} {a_i})$。

$q$ 次询问,每次给定 $l,r$,询问最小的 $k$ 使得 $f^k (l, r) = (1, n)$,无解输出 $-1$。

\subsection{题解}

首先两个十分显然的性质:

\begin{itemize}
	\item 如果某次操作把区间变成了 $(1, n)$,那么无论再操作多少次这个区间都是 $(1, n)$;
	\item 状态数是 $O(n^2)$ 的。
\end{itemize}

这引导我们想到,如果能求得 $f^k(l, r)$ 在 $k \ge n^2$ 时的结果,就能判定是否有解,同时也可以利用二分之类的方法求得答案。

Key Observation 1: $f(l, r) = \bigcup_{i=l}^{r-1}{f(i, i+1)}$。

证明:考虑归纳,则只需证明:$[l_1, r_1] \cup [l_2, r_2] = [l, r], [l_1, r_1] \cap [l_2, r_2] \ne \varnothing$,则 $f(l, r) = f(l_1, r_1) \cup f(l_2, r_2)$。而这是显然的。

Key Observation 2: $f^k(l, r) = \bigcup_{i=l}^{r-1}{f^k(i, i+1)}$。

证明:考虑上一页中结论,每次增加 $k$ 相邻两个区间仍然总是有交。

$$
\begin{aligned}
	&f^k(l,r) \\
	=& f^k([l_1, r_1] \cup [l_2, r_2]) \\
	=& f(f^{k - 1}(l_1, r_1) \cup f^{k - 1}(l_2, r_2)) \\
	=& f(f^{k - 1}(l_1, r_1)) \cup f(f^{k - 1}(l_2, r_2)) \\
	=& f^{k}(l_1, r_1) \cup f^{k}(l_2, r_2) \\
\end{aligned}
$$

然后我们发现到最后相邻两项区间还是有交,因此我们最终区间到左、右端点就是这些区间左、右节点的极值。这允许我们通过维护 $[i, i+1]$ 的信息,并利用 st 表求得任意区间的结果。

我们令 $F/G_{k, j, i}$ 为 $f^k(j, j+2^i - 1)$ 的左、右端点。那么对于 $i$ 这维的转移,我们有:

$$
\begin{aligned}
	F_{k,j,i} &= \min(F_{k,j,i - 1}, F_{k,j + 2^{i-1}, i - 1}) \\
	G_{k,j,i} &= \max(G_{k,j,i - 1}, G_{k,j + 2^{i-1}, i - 1})
\end{aligned}
$$

对于 $k$ 这一维,我们有:

$$
\begin{aligned}
	&f^k(l,r) = f(f^{k - 1}(l_1, r_1) \cup f^{k - 1}(l_2, r_2)) \\
\end{aligned}
$$

那么我们知道 $f^k(l, r)$ 的左右端点分别是:

$$
\begin{aligned}
	&\min(F_{k, l, lg}, F_{k,r - 2^{lg} + 1,lg}) \\
	&\max(G_{k, l, lg}, G_{k,r - 2^{lg} + 1,lg})
\end{aligned}
$$

至此,预处理后我们能够在 $O(1)$ 时间内求解 $f^k(l, r)$。二分或倍增即可求得答案。

\subsection{代码}

代码实现把 $F$ 和 $G$ 放在了同一个数组里来卡常。

\begin{verbatim}
#pragma GCC optimize("Ofast")

#include <bits/stdc++.h>

const int MAX = 1e5 + 5;
const int LG = 35;
const int MAXX = 37;

int n, q, l, r;
int a[MAX], lg2[MAX], f[MAXX][MAX][20][3];

inline int read()
{
	char c=getchar();int x=0;bool f=0;
	for(;!isdigit(c);c=getchar())f^=!(c^45);
	for(;isdigit(c);c=getchar())x=(x<<1)+(x<<3)+(c^48);
	if(f)x=-x;return x;
}

inline int min(int a, int b) { return a < b ? a : b; }
inline int max(int a, int b) { return a > b ? a : b; }
void init(int k)
{
	for (int i = 1; (1 << i) < n; ++i)
	{
		for (int j = 1; j + (1 << i) <= n; ++j)
		{
			f[k][j][i][0] = min(f[k][j][i - 1][0], f[k][j + (1 << (i - 1))][i - 1][0]);
			f[k][j][i][1] = max(f[k][j][i - 1][1], f[k][j + (1 << (i - 1))][i - 1][1]);
		}
	}
}

int getl(int l, int r, int k)
{
	int lg = lg2[r - l + 1];
	return min(f[k][l][lg][0], f[k][r - (1 << lg) + 1][lg][0]);
}

int getr(int l, int r, int k)
{
	int lg = lg2[r - l + 1];
	return max(f[k][l][lg][1], f[k][r - (1 << lg) + 1][lg][1]);
}

void solve()
{
	l = read(); r = read(); long long res = 0;
	if (l == 1 and r == n) { printf("0\n"); return ; }
	else if (l == r) { printf("-1\n"); return ; }
	int _l = getl(l, r - 1, LG), _r = getr(l, r - 1, LG);
	if (_l != 1 or _r != n) { printf("-1\n"); return ; }
	for (int i = LG - 1; i >= 0; --i)
	{
		_l = getl(l, r - 1, i), _r = getr(l, r - 1, i);
		if (_l != 1 or _r != n) { res += (1ll << i); l = _l; r = _r; }
	}
	_l = getl(l, r - 1, 0), _r = getr(l, r - 1, 0);
	if (_l == 1 and _r == n) { printf("%lld\n", res + 1); } else { printf("-1\n"); }
}

int main()
{
	n = read(); q = read();
	for (int i = 1; i <= n; ++i) { a[i] = read(); }
	for (int i = 2; i <= n; ++i) { lg2[i] = lg2[i >> 1] + 1; }
	for (int i = 1; i <  n; ++i) { f[0][i][0][0] = min(a[i], a[i + 1]); f[0][i][0][1] = max(a[i], a[i + 1]); }
	init(0);
	for (int i = 1; i <= LG; ++i)
	{
		for (int j = 1; j <  n; ++j)
		{
			f[i][j][0][0] = getl(f[i - 1][j][0][0], f[i - 1][j][0][1] - 1, i - 1);
			f[i][j][0][1] = getr(f[i - 1][j][0][0], f[i - 1][j][0][1] - 1, i - 1);
		}
		init(i);
	}
	while (q--) { solve(); }
	return 0;
}
\end{verbatim}
\end{document}