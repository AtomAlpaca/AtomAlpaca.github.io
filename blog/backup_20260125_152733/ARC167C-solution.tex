\documentclass{article}
\usepackage[UTF8]{ctex}
\usepackage{hyperref}
\usepackage{fancyvrb}
\usepackage{amsmath}
\hypersetup {
	colorlinks = true,
	linkcolor  = black,
}

\newcommand{\[}{\begin{align*}}
	\newcommand{\]}{\end{align*}}
\renewcommand{\par}[1]{\paragraph{#1}}


\author {AtomAlpaca}
\title {「题解」ARC167C MST on Line++}

\begin{document}
	
	{
		\setcounter{tocdepth}{3}
		\tableofcontents
	}
	\subsection{题意}
	
	给定正整数 \(n,K\) 和一个长度为 \(n\) 的序列 \(A\)。对于一个 \(1\sim n\) 的排列 \(P\),我们定义 \(f(P)\) 为以下问题的答案:
	
	\begin{quote}
		给一个 \(n\) 个点的无向带权图,对于两点 \(i<j\),当且仅当 \(j-i\le K\) 时,它们之间有边,边权为 \(\max(A_{P_i},A_{P_j})\)。\\
		求这个图的最小生成树边权和。
	\end{quote}
	
	对于所有可能的排列 \(P\),求出它们的 \(f(P)\) 之和,答案对 \(998\,244\,353\) 取模。
	
	\(1\le K< N\le 5000\),\(1\le A_i \le 10^9\)。
	
	奇妙深刻数数题。
	
	\subsection{题解}
	
	首先发现因为要枚举排列,\(A\) 的顺序是无关紧要的,可以升序排序处理。
	
	考虑拆贡献,考虑对于每一个 \(A_i\) 计算它在所有情况中被选的次数 \(f_i\)。答案就是 \(\sum_{i}^{n}{f_i  A_i}\)。
	
	但是这样还是不好求,我们考虑用小于等于 \(A_i\) 的边权被选的次数减去小于等于 \(A_{i - 1}\) 的边权的被选次数。这相当于构造了这样一个问题:
	
	对于两点 \(i<j\),当且仅当 \(j-i\le K\) 且 \(\max(A_{P_i},A_{P_j}) \le A_x\) 时,它们之间有边。求最多选择多少条边使得构成的图没有环,对于所有排列 \(P\) 求和。
	
	因为取 \(\max\),所有 \(A_{P_i} > x\) 的 \(i\) 都不会被选,被选的只有小于 \(A_x\) 的所有位置。
	
	再考虑 \(j - i \le K\) 这个条件怎么做。我们这里不妨对于一个所有选出来的数的集合有序 \(Q\),考虑钦定 \(Q_j - Q_{j - 1} = K\),这样的方案一共有 \(\binom{n-k}{x-1}\) 种,小于等于 \(K\) 的情况就有 \(\sum_{i = 1}^{K}{\binom{n-k}{x-1}}\) 种。一共有 \(x-1\) 个这样的位置,那么贡献总和就是 \((x - 1)\sum_{i = 1}^{K}{\binom{n-i}{x-1}}\)
	
	又考虑和一个 \(Q\) 对应的排列的排列方式应该是选出来的 \(x\) 个随便排,剩下的 \(n - x\) 个随便排并在一起,方案数应该是 \(x!(n - x)!\)。
	
	所以上述问题的答案就是 \(x!(n - x)!(x - 1)\sum_{i = 1}^{K}{\binom{n-i}{x-1}}\)。相邻的两项相减就能得到 \(f\)。
	
	\subsection{代码}
	
	\begin{verbatim}
#include <bits/stdc++.h>

typedef long long ll;
const int MAX = 5005;
const int MOD = 998244353;

int n, k; ll ans;
ll frc[MAX], ifrc[MAX], g[MAX], a[MAX];

ll qp(ll a, ll x)
{
	ll res = 1;
	while (x) { if (x & 1) { res = res * a % MOD; } x >>= 1; a = a * a % MOD; }
	return res;
}

void init(int x)
{
	frc[0] = ifrc[0] = 1;
	for (int i = 1; i <= x; ++i) { frc[i] = frc[i - 1] * i % MOD; } ifrc[x] = qp(frc[x], MOD - 2);
	for (int i = x - 1; i >= 1; --i) { ifrc[i] = ifrc[i + 1] * (i + 1) % MOD; }
}

ll C(ll x, ll y) { if (x < y) { return 0; } return frc[x] * ifrc[y] % MOD * ifrc[x - y] % MOD; }

int main()
{
	scanf("%d%d", &n, &k); init(5000);
	for (int i = 1; i <= n; ++i) { scanf("%lld", &a[i]); }
	std::sort(a + 1, a + n + 1);
	for (int i = 1; i <= n; ++i)
	{
		ll tmp = 0;
		g[i] = frc[i] * frc[n - i] % MOD * (i - 1) % MOD;
		for (int j = 1; j <= k; ++j)
		{
			tmp = (tmp + C(n - j, i - 1)) % MOD;
		}
		g[i] = g[i] * tmp % MOD;
	}
	for (int i = 1; i <= n; ++i) { ans = (ans + (g[i] - g[i - 1] + MOD) % MOD * a[i] % MOD) % MOD; }
	printf("%lld", ans);
	return 0;
}
	\end{verbatim}
	
\end{document}