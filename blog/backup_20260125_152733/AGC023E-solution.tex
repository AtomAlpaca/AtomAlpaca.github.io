\documentclass{article}
\usepackage[UTF8]{ctex}
\usepackage{hyperref}
\usepackage{fancyvrb}
\usepackage{amsmath}

\author {AtomAlpaca}
\title {「题解」AGC023E Inversions}

\begin{document}
\tableofcontents
\subsection{题意}

给定一个长度为 $n$ 的数组 $a$,问对于所有满足 $\forall i, p_i \le a_i $ 的排列 $p$ 的逆序对个数和。

\subsection{题解}

首先令 $b$ 为 $a$ 从小到大排序的结果,$rk_i$ 表示 $a_i$ 的排名,$id_i$ 表示排名为 $i$ 的数在 $a$ 中的下标。

那么所有符合条件的排列个数为 $cnt = \prod_{i=1}^{n}{a_i-rk_i+1}$。因为我们从小到大考虑每个排名 $x$,那么它前面已经被选了 $x - 1$ 个 $a_{id_x}$ 范围内的数,那么这个位置能选择的数就有 $a_{id_x} - x + 1$ 个。

那么接下来我们依然从小到大考虑排名。假设我们考虑到排名 $i$,那么对于每个 $j < i$,我们分情况讨论:

首先如果 $id_i > id_j$,那么我们 $a_{id_i}$ 比 $a_{id_j}$ 大的部分并不会产生贡献,我们直接扔掉不考虑。那么我们把 $a_{id_i}$ 削到和 $a_{id_j}$ 相等,此时两个位置上选择方案有 $(a_{id_j} - j + 1)(a_{id_j} - j)$ 种,其中只有一半是形成逆序对的。

然后我们发现如果强行让它的范围缩小,那么对所有满足 $j < k < i$ 的 $k$,我们可以选择的数其实是减少了一个。于是我们得到这一对点的贡献是:

$$
\frac{(a_{id_j} - j + 1)(a_{id_j} - j)}{2} \frac{cnt}{(a_{id_i} - i + 1)(a_{id_j} - j + 1)} \prod_{k = j + 1}^{i - 1}{\frac{a_{id_k} - k}{a_{id_k} - k + 1}}
$$

如果 $id_i < id_j$,我们考虑直接求产生顺序对的数量,这和上面的式子是一样的。只需要用总排列数减去即可。

考虑维护最后一项的前缀积是可以轻松做到 $O(n^2)$ 的,但是这并不足够通过此题。我们考虑把上面的式子化简下。

$$
\begin{aligned}
	&\frac{(a_{id_j} - j + 1)(a_{id_j} - j)}{2} \frac{cnt}{(a_{id_i} - i + 1)(a_{id_j} - j + 1)} \prod_{k = j + 1}^{i - 1}{\frac{a_{id_k} - k}{a_{id_k} - k + 1}}
	\\
	=&\frac{cnt}{2(a_{id_i} - i + 1)} (a_{id_j} - j)\prod_{k = j + 1}^{i - 1}{\frac{a_{id_k} - k}{a_{id_k} - k + 1}}
\end{aligned}
$$
最前面的一项是只和 $i$ 有关的,我们只需要维护后面的两项。

考虑我们每次枚举 $i$ 之后对后面的影响其实就是把最后一项乘了个 $\dfrac{a_{id_i} - i}{a_{id_i} - i + 1}$,然后要多考虑一个 $i$。那么我们考虑用一个支持全局乘、单点加、区间和的线段树,每个位置 $i$ 记录 $id_i$ 的后面两项的值。

这样我们的答案就是 $qry(1, id_i) + p \times cnt - qry(id_i, n)$,其中 $p$ 是 $j > i$ 的点数。每次统计完 $i$ 的答案全局乘 $\dfrac{a_{id_i} - i}{a_{id_i} - i + 1}$ 单点加 $a_{id_i} = j$ 即可。

另外,我们用线段树只能得到所有 $j > i$ 的答案,并不知道有多少个,因此要类似二维数点地开一个树状数组维护。

这样我们就可以以 $O(n \log n)$ 的时间复杂度解决这个问题。

\subsection{代码}

\begin{verbatim}
#include <bits/stdc++.h>

typedef long long ll;
const int MAX = 2e5 + 5;
const int MOD = 1e9 + 7;

ll n, tot, ans;
ll a[MAX], rk[MAX], id[MAX];

struct N { ll x, y; } b[MAX];
bool cmp(N a, N b) { return (a.x == b.x ? a.y < b.y : a.x < b.x); }

ll qp(ll a, ll x)
{
	ll res = 1;
	while (x) { if (x & 1) { res = res * a % MOD; } a = a * a % MOD; x >>= 1; }
	return res;
}

ll inv(ll x) { return qp(x, MOD - 2); }

struct BIT
{
	int t[MAX];
	inline int lbt(int x) { return x & -x; }
	void mdf(int x, int c) { while (x <= n) { t[x] += c; x += lbt(x); } }
	int qry(int x) { int res = 0; while (x) { res += t[x]; x -= lbt(x); } return res; }
} bt;

struct SGT
{
	ll st[MAX << 2], tg[MAX << 2];
	void init() { for (int i = 1; i <= 4 * n; ++i) { tg[i] = 1; } }
	inline void pd(int x)
	{
		st[x << 1] = st[x << 1] * tg[x] % MOD; st[x << 1 | 1] = st[x << 1 | 1] * tg[x] % MOD;
		tg[x << 1] = tg[x << 1] * tg[x] % MOD; tg[x << 1 | 1] = tg[x << 1 | 1] * tg[x] % MOD;
		tg[x] = 1;
	}
	inline void pu(int x) { st[x] = (st[x << 1] + st[x << 1 | 1]) % MOD; }
	void mdf0(ll v) { st[1] = st[1] * v % MOD; tg[1] = tg[1] * v % MOD; }
	void mdf1(int l, int r, int s, ll c, int x)
	{
		if (l == r and l == s) { st[x] += c; return ; }
		pd(x); int k = l + ((r - l) >> 1);
		if (k >= s) { mdf1(l, k, s, c, x << 1); }
		else { mdf1(k + 1, r, s, c, x << 1 | 1); }
		pu(x);
	}
	ll qry(int l, int r, int s, int t, int x)
	{
		if (l >= s and r <= t) { return st[x]; }
		pd(x); int k = l + ((r - l) >> 1); ll res = 0;
		if (k >= s) { res = (res + qry(l, k, s, t, x << 1)) % MOD; }
		if (k <  t) { res = (res + qry(k + 1, r, s, t, x << 1 | 1)) % MOD; }
		return res;
	}
} st;

int main()
{
	scanf("%lld", &n); tot = 1; st.init();
	for (int i = 1; i <= n; ++i) { scanf("%lld", &a[i]); b[i] = {a[i], i}; }
	std::sort(b + 1, b + n + 1, cmp);
	for (int i = 1; i <= n; ++i) { id[i] = b[i].y; rk[id[i]] = i; }
	for (int i = 1; i <= n; ++i)
	{
		if (b[i].x - i + 1 <= 0) { printf("0"); return 0; }
		tot = tot * (b[i].x - i + 1) % MOD;
	}
	for (int i = 1; i <= n; ++i)
	{
		ll res1 = st.qry(1, n, 1, id[i], 1), res2 = st.qry(1, n, id[i], n, 1), cnt = bt.qry(n) - bt.qry(id[i]);
		ll tmp = inv(2) * inv(a[id[i]] - i + 1) % MOD * tot % MOD;
		ans = (ans + tmp * (res1 - res2 + MOD) % MOD) % MOD;
		ans = (ans + cnt * tot % MOD) % MOD;
		st.mdf0((a[id[i]] - i) * inv(a[id[i]] - i + 1) % MOD);
		st.mdf1(1, n, id[i], a[id[i]] - i, 1);
		bt.mdf(id[i], 1);
	}
	printf("%lld", ans);
	return 0;
}
\end{verbatim}
\end{document}