\documentclass{article}
\usepackage[UTF8]{ctex}
\usepackage{hyperref}
\usepackage{graphicx}

\author {AtomAlpaca}
\title  {「算法」wqs 二分略解}
\begin{document}
	\tableofcontents
	\section{简介} 
	wqs 二分,最早在 2012 年由王钦石在其集训队论文中提出,后因 IOI 2016 中的题目 Aliens 使用了这一算法而闻名,因此这一算法也被称为“Alien's trick”
	
	wqs 二分能解决带有以下特征的题目:
	
	- 求恰好取得 $k$ 个物品时的最优答案 $f(k)$
	- $f(k)$ 较难直接求得,但容易求得无限制时的最佳答案
	- $f$ 具有凸性
	
	\section{原理} 
	
	我们把所有 $(x, f(x))$ 画出来,由于 $f$ 有凸性,它们构成了一个凸包。我们不妨只讨论上凸,也就是斜率递减的情况,其余的情况是类似的。
	
	我们考虑二分一条直线的斜率 $p$,然后用这条直线去切这个凸包。因为 $f(x)$ 有凸性,切到的点关于 $p$ 是单调的,因此我们可以通过不断调整 $p$ 使得这个直线切在 $(k, f(k))$ 上,然后即可快速求得答案。现在问题转化为如何快速求得这条直线在凸包上切在哪个位置。
	
	\includegraphics[width=0.80\textwidth]{https://cdn.luogu.com.cn/upload/image_hosting/1r9cxsrx.png}
	
	由于我们并不知道凸包的具体结构,考虑继续转化问题。考虑在凸包上每个点,切到的点一定使得这条直线过这个点时在 $y$ 轴上的截距最大。不妨设直线过 $(x, f(x))$ 上,则直线的截距为 $b = f(x) - px$,我们要最大化这个东西。
	
	由于 $f(x)$ 是恰好选择 $x$ 个时的最大收益,那么后面的 $- px$ 一项可以看作是每次选择都要额外扣除 $p$ 的权值。也就是说,我们实际要做的就是给每个物品扣除 $p$ 的权值,然后在没有约束条件的情况下求得最优时选择的数量,此时得到的就是斜率为 $p$ 交到的点的横坐标,此时求得的最佳答案只需要把调整的总共 $px$ 权值加回去就能得到答案。
	
	但有些情况下,凸包上会有连续几个点共线的情况(或者说某些 $(x, f(x))$ 没有成为凸包的顶点而是在凸包的边上),这时被夹在中间的点无论如何都无法成为切点。这种情况怎么处理呢?
	
	我们考虑切线与凸包上多点共线这种情况意味着什么:
	
	- 连续几个物品的权值相等;
	- 这几个物品调整后权值恰好都为 $0$。
	
	这时我们发现这几个物品在调整是“无意义”的,可选可不选;而因此这几个位置在调整后求出的答案是相同的,于是两侧的点的答案可以直接拿来用,只需要加上正确的 $kp$ 即可。
	
	代码实现上,由于我们很难判断是否有共线的情况出现,我们可以钦定“无意义”的的物品都选中,并且每次选择的物品数大于 $k$ 就更新答案。注意这里更新答案要加的是 $kp$ 而不是当前点 $x$ 对应的 $px$。
	
	\section{例题} 
	\subsection{最小度限制生成树} 
	\subsubsection{题意} 
	给定一张带权无向联通图,求得一个生成树,满足节点 $s$ 度数恰好为 $k$,且边权总和最小。
	\href{https://www.luogu.com.cn/problem/P5633}{link}
	\subsubsection{题解}
	我们首先证明问题具有凸性。考虑一个没有实际实现意义的算法,假设我们已经求得了 $s$ 度数为 $k - 1$ 的最小生成树,那么我们每次选择 $s$ 连出去的,没有构成生成树的边中选择一个,设其权值为 $w_1$,把它强行加到树上,这时树上出现了一个环,我们再删掉这个环上除刚刚加入的这条边外权值最大的一条边 $w_2$,每次都选择 $w_1 - w_2$ 最小的一个。能选的边是不断减少的,如果这次能选择这对边,那上一次一定也可以,因此不难发现每次的 $w_1 - w_2$ 是不增的,因此发现问题具有凸性。
	
	因此我们可以使用 wqs 二分。每次将所有和 $s$ 相连的边权值减去 $p$,然后把所有边一起跑一遍 Kruskal,根据得到的生成树中 $s$ 的度数继续二分即可。
	
	\begin{verbatim}
#include <bits/stdc++.h>

typedef long long ll;
const int MAX = 5e4 + 5;
const int MAXX = 5e5 + 5;

int n, m, s, K, k, l, r, cnt, _cnt, mn = MAX, mx = -MAX; ll res, ans;
int f[MAX];

struct E { int u, v, w; } e[MAXX];
inline bool cmp(E a, E b) { return a.w < b.w; }

int find(int x) { return f[x] == x ? x : f[x] = find(f[x]); }

void init()
{
	_cnt = cnt = res = 0;
	std::sort(e + 1, e + m + 1, cmp);
	for (int i = 1; i <= n; ++i) { f[i] = i; }
}

void solve()
{
	init();
	for (int i = 1; i <= m and _cnt < n - 1; ++i)
	{
		int fu = find(e[i].u), fv = find(e[i].v);
		if (fu == fv) { continue; }
		++_cnt; res += e[i].w;
		cnt += (e[i].u == s or e[i].v == s); f[fv] = fu;
	}
}

int main()
{
	scanf("%d%d%d%d", &n, &m, &s, &K);
	for (int i = 1; i <= m; ++i) { scanf("%d%d%d", &e[i].u, &e[i].v, &e[i].w); r = std::max(r, e[i].w); }
	l = -r;
	while (l <= r)
	{
		k = l + ((r - l) >> 1);
		for (int i = 1; i <= m; ++i) { if (e[i].u == s or e[i].v == s) { e[i].w += k; } }
		solve(); if (cnt < K) { r = k - 1; } else { ans = res - k * K; l = k + 1; }
		mx = std::max(mx, cnt), mn = std::min(mn, cnt);
		for (int i = 1; i <= m; ++i) { if (e[i].u == s or e[i].v == s) { e[i].w -= k; } }
	}
	if (K > mx or K < mn) { printf("Impossible"); return 0; }
	printf("%lld", ans);
}
	\end{verbatim}
	
	\subsection{[国家集训队] Tree I } 
	\subsubsection{题意}
	给出一个无向带权连通图,每条边是黑色或白色。求一棵最小权的恰好有 $need$ 条白色边的生成树。
	\href{https://www.luogu.com.cn/problem/P2619}{link}
	\subsubsection{题解}
	
	凸性的证明思路和最小度限制生成树类似。首先对于每个白边 $(u, v, w)$,考虑新增一条黑边 $(u, v, +\infty)$,防止白边过少时无法得到生成树。考虑当前有 $k - 1$ 条白边的生成树,现在要强行加进去一个白边,设其权值为 $w_1$,这使得原来的生成树上多出来一个环,我们选这个环上所有黑边中权值最大的一个删掉,设其权值为 $w_2$。我们每次要选择加入后构成的环中有黑边,且权值增量 $w = w_1 - w_2$ 最小的一个白边。不难发现这两个限制随着白边的加入都是越来越紧的——第 $k$ 次未加入生成树的白边第 $k + 1$ 次可能加入了,第 $k$ 次存在的黑边第 $k + 1$ 次可能不存在。
	
	因此如果第 $k + 1$ 次加入时的增量小于 $k$ 次,那一定可以交换 $k, k + 1$ 次的选边方案使得第 $k$ 次更优,因此加入白边后的权值增量不增,故该问题有凸性,可以用 wqs 二分解决。
	
	于是二分一个增量,每次将所有白边权值加上这个增量,再和黑边一起跑最小生成树,根据最小生成树中白边的数量调整增量即可。
	
	\subsubsection{代码}
\begin{verbatim}

#include <bits/stdc++.h>

using std::cin;
using std::cout;

const int MAX = 1e5 + 5;

int n, m, k, u, v, w, c, cnt, res, ans;
int f[MAX];
struct E { int u, v, w, c; } e[MAX];

int find(int x) { if (f[x] == x) { return x; } return f[x] = find(f[x]); }
bool cmp(E a, E b) { if (a.w == b.w) { return a.c < b.c; } return a.w < b.w; }

void check(int x)
{
	res = cnt = 0;
	for (int i = 1; i <= m; ++i) { if (e[i].c == 0) { e[i].w += x; } }
	for (int i = 1; i <= n; ++i) { f[i] = i; }
	std::sort(e + 1, e + m + 1, cmp);
	for (int i = 1, tmp = 0; i <= m and tmp != n - 1; ++i)
	{
		int fu = find(e[i].u), fv = find(e[i].v);
		if (fu == fv) { continue; }
		f[fv] = fu; if (e[i].c == 0) { ++cnt; } res += e[i].w; ++tmp;
	}
	for (int i = 1; i <= m; ++i) { if (e[i].c == 0) { e[i].w -= x; } }
}

int main()
{
	cin >> n >> m >> k;
	for (int i = 1; i <= m; ++i)
	{
		cin >> u >> v >> w >> c;
		e[i] = {u + 1, v + 1, w, c};
	}
	int l = -111, r = 111, mid = 0;
	while (l <= r)
	{
		mid = (l + r) >> 1;
		check(mid);
		if (cnt < k) { r = mid - 1; } else { l = mid + 1; ans = res - k * mid; }
	}
	cout << ans;
	return 0;
}
	\end{verbatim}
\end{document}
