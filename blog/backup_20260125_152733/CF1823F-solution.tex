\documentclass{article}
\usepackage[UTF8]{ctex}
\usepackage{hyperref}
\usepackage{fancyvrb}
\usepackage{amsmath}
\hypersetup {
	colorlinks = true,
	linkcolor  = black,
}

\newcommand{\[}{\begin{align*}}
\newcommand{\]}{\end{align*}}
\renewcommand{\par}[1]{\paragraph{#1}}


\author {AtomAlpaca}
\title {「题解」CF1823F Random Walk}
\begin{document}

\subsection{题目}

简述题意:给定一棵树和两个节点 \(S, T\),从 \(S\)
出发,在树上随机游走直到走到 \(T\) 为止,求出每个节点的期望经过次数。

感觉树上做法很奇妙,于是写一篇题解!

\subsection{题解}\label{ux9898ux89e3}

令 \(d_u\) 为节点 \(u\) 的度数,\(f_u\) 为节点 \(u\)
的期望经过次数。每次移动时,会从当前节点的所有节点中等概率选择一个节点,因此到达每个相邻节点的概率都是
\(\dfrac{1}{d_u}\);从被到达节点的角度考虑,就有
\(f_u = \sum_{(u, v)\in E} { \dfrac{f_v}{d_v} }\)。

特别地,节点不能从 \(T\) 转移过来,因为到达 \(T\) 后游走就会停止;因为从
\(S\) 节点出发,\(f_S\) 初始值为 \(1\)。整理一下:

\[
f_u =
\left\{ 
\begin{aligned} 
& 1, &if \ u=T 
\\
&[u=S] + \sum_{(u, v)\in E,v\ne T} { \dfrac{f_v}{d_v} }, &otherwise
\end{aligned}
\right.
\]

然而我们发现转移是有环的,不能直接
dp。我们固然可以使用高斯消元解决这个方程组,这也是\href{https://www.luogu.com.cn/problem/P3232}{有环图上随机游走问题}的最好解决办法,但是
\(O(n^3)\) 的复杂度无法通过此题。

考虑到这道题的图是一棵树,我们可以考虑把式子化成 \(f_u=g_u f_{fa} +c_u\)
的形式。由于树根是没有父亲的,我们能够直接求出树根的期望经过次数,然后
dfs 下去求得每个节点的经过次数。

\[
\begin{aligned} 
f_u &= [u=S] + \sum_{(u, v)\in E,v\ne T} { \dfrac{f_v}{d_v} } \\
&= [u=S] + \dfrac{f_{fa}}{d_{fa}} + \sum_{(u, v)\in E,v\ne T,fa} { \dfrac{f_v}{d_v}} \\
&= [u=S] + \dfrac{f_{fa}}{d_{fa}} + \sum_{(u, v)\in E,v\ne T,fa} { \dfrac{g_v f_u + c_v}{d_v}} \\
&= [u=S] + \dfrac{f_{fa}}{d_{fa}} + \sum_{(u, v)\in E,v\ne T,a} { \dfrac{g_v f_u}{d_v}} + \sum_{(u, v)\in E,v\ne T,fa} { \dfrac{c_v}{d_v}}\\
\end{aligned}
\] 移项得到: \[
\begin{aligned} 
(1-\sum_{(u, v)\in E,v\ne T,fa}{\dfrac{g_v}{d_v}}) f_u &= [u=S] + \dfrac{f_{fa}}{d_{fa}} + \sum_{(u, v)\in E,v\ne T, fa} { \dfrac{c_v}{d_v}}
\end{aligned}
\]

\[
  f_u &= \dfrac{1}{d_{fa}(1-\sum_{(u, v)\in E,v\ne T,fa}{\dfrac{g_v}{d_v}})} f_{fa} + \dfrac{[u=S]+\sum_{(u, v)\in E,v\ne T,fa} { \dfrac{c_v}{d_v}}}{(1-\sum_{(u, v)\in E,v\ne T, fa}{\dfrac{g_v}{d_v}})}
  \\
f_u &= g_u f_{fa} + c_u
\]

于是我们得到了 \(g, c\)
的转移方程,而且这个转移是仅依赖自己的子节点的!于是我们可以先一遍 dfs
求出每个节点的 \(g, c\),再求出 \(f\)。

\subsection{代码}

\begin{verbatim}
#include <bits/stdc++.h>

typedef long long ll;

const int MOD = 998244353;
const int MAX = 2e5 + 5;

int n, S, T, u, v, tot;
int h[MAX], f[MAX], c[MAX], g[MAX], d[MAX];

struct E { int v, x; } e[MAX << 2];
void add(int u, int v) { e[++tot] = {v, h[u]}; h[u] = tot; e[++tot] = {u, h[v]}; h[v] = tot; }

inline int qp(int a, int x)
{
	int res = 1;
	while (x) { if (x & 1) { res = 1ll * res * a % MOD; } a = 1ll * a * a % MOD; x >>= 1; }
	return res;
}

void dfs(int u, int fa)
{
	int _g = 0, _c = (u == S);
	for (int i = h[u]; i; i = e[i].x)
	{
		int v = e[i].v; if (v == fa) { continue; }
		dfs(v, u);
		_g = (_g + 1ll * g[v] * qp(d[v], MOD - 2) % MOD) % MOD;
		_c = (_c + 1ll * c[v] * qp(d[v], MOD - 2) % MOD) % MOD;
	}
	_g = qp((1 - _g + MOD), MOD - 2);
	_c = 1ll * _c * _g % MOD;
	_g = 1ll * _g * qp(d[fa], MOD - 2) % MOD;
	g[u] = _g; c[u] = _c;
}

void _dfs(int u, int fa)
{
	f[u] = (1ll * g[u] * f[fa] % MOD + c[u]) % MOD;
	for (int i = h[u]; i; i = e[i].x)
	{
		int v = e[i].v; if (v == fa) { continue; }
		_dfs(v, u);
	}
}

int main()
{
	scanf("%d%d%d", &n, &S, &T);
	for (int i = 1; i < n; ++i) { scanf("%d%d", &u, &v); add(u, v); ++d[u]; ++d[v]; }
	dfs(T, 0); f[T] = 1; _dfs(T, 0); f[T] = 1;
	for (int i = 1; i <= n; ++i) { printf("%d ", f[i]); }
	return 0;
}
\end{verbatim}

\end{document}
