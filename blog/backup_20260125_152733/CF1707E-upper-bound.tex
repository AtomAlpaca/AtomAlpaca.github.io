\documentclass{article}
\usepackage[UTF8]{ctex}
\usepackage{hyperref}
\usepackage{fancyvrb}
\usepackage{amsmath}
\hypersetup {
	colorlinks = true,
}

\newcommand{\[}{\begin{align*}}
	\newcommand{\]}{\end{align*}}
\renewcommand{\par}[1]{\paragraph{#1}}

\author {AtomAlpaca}
\title {「数学」遥远的记忆:CF1707E 的答案上界及其证明}

\begin{document}
	\tableofcontents
	\subsection{前言}
当成北京集训讲这道题的时候也有人问了答案上界的问题,当时一时间没能答上来,趁着同学讲题的时间想出了证明,于是返场讲了一下,掌声雷动。两年后的今天因为\href{https://www.luogu.com.cn/discuss/1117450}{这篇帖子}再次尝试寻找答案的上界,曾经能被我十分钟想出来的证明现在花了我将近一个小时,令人感叹。想念那年北京的第一场雪。

题面及其解法请移步\href{./CF1707E-solution.html}{这篇博客}。
	\subsection{证明}
设一共有 $S$ 次操作
	
我们把操作分为两个 stage,第二个 stage 满足:
	
$\exists k < S, \forall S > i > k, [l_{i}, r_{i}] \subset [l_{i + 1}, r_{i + 1}]$
	
不难看出这个 stage 一定存在。我们把剩下的操作全划分进 stage $1$

首先几个 observision:
	
Observision $1.$ 任何时间,如果区间长度为 $1$,则一定无解
	
Proof: 显然。
	
Observision $2.$ 任何时间,如果一个区间是之前到过某个区间的子区间,则无解。形式化地说,$[l_{k + 1}, r_{k + 1}] \subseteq [l_k, r_k]$ 时一定无解。
	
Proof. 产生循环。
	
得到推论,stage 2 从长度为 $2$ 的区间开始,最多经过 $n - 2$ 次操作得到 $[1, n]$。
	
Observision $3.$,如果 $\exists k\ s.t. [l_{k - 1}, r-{k - 1}] \subseteq [l_k, r_k]$,则有 $\forall S > i > k, [l_i, r_i] \subseteq [l_{i - 1}, r_{i - 1}]$,也即一个区间开始向两侧“扩张”,就会一直扩张下去。
	
Proof. 扩张之后最小值不增最大值不减。
	
也就是说一旦产生了一次“扩张”就会直接进入 stage2
	
接下来再考虑 stage 1。这时我们得到了一个问题:给出一个长度为 $n$ 的线段,让你往上面不断放区间,要求不能是之前放过区间的子集,也不能是之前放过区间的超集,问最多能放下多少个(我们不考虑能不能构造出对应的 $a$ 数组,因为我不会。)。

显然最优解是一直放长度为 $2$ 的区间,一共能放 $n - 1$ 次,然后一定会产生“扩张”进入 stage 2。我们放的第一个区间是题目给出的,放的最后一个区间就是 stage2 的第一个区间,因此最大操作次数为 $2n - 4$ 次。

当 $n=3,a=[2,3,1],f(1,2)$ 就能卡到上界,因此该结果无法继续改进。
\end{document}