\documentclass{article}
\usepackage[UTF8]{ctex}
\usepackage{hyperref}
\usepackage{fancyvrb}
\usepackage{amsmath}
\hypersetup {
	colorlinks = true,
	linkcolor  = black,
}

\newcommand{\[}{\begin{align*}}
	\newcommand{\]}{\end{align*}}
\renewcommand{\par}[1]{\paragraph{#1}}

\author {AtomAlpaca}
\title {「信创」如何给 ncnn 做 riscv 优化}

\begin{document}
	\tableofcontents
	\section{起因}
	起因是我参加了前段时间的\href{Riscv 黑客松}[https://www.bilibili.com/video/BV1NpdeYHETU/],入门了 RISC-V。然后突然有一天 mizu-bai 找到我发了一段聊天记录。
	\begin{verbatim}
		nihui:github.com/atomalpaca
		nihui:这头像怎么和 mizu 这么像
		小小跑:这头像怎么和 mizu 这么像
	\end{verbatim}
	
	然后我问这是什么群,mizu 姐姐说是 ncnn 的开发者群,给 ncnn 交个 pr 就能进,快来玩。
	
	然后我看了一圈 issues,看了一圈代码,我说我不知道干什么啊,mizu 姐姐说有很多算子在 riscv 上没有优化你可以折腾一下。
	
	\section{配置环境}
	\subsection{交叉编译工具链}
	首先我们需要一套 Riscv 的交叉编译工具链。riscv 给出的构建默认不包含 v 拓展,于是我们需要自己编译一份出来。
	
	另外我们还需要支持 xtheadvector,这在 toolchain 中默认的 gcc 14 中是不支持的,所以我们要手动拉一份 gcc 15。
	
	P.S. 写这篇文章的时候 gcc16 出来了,读者可以尝试一下()
	
	\begin{verbatim}	
git clone https://github.com/riscv-collab/riscv-gnu-toolchain
cd riscv-gnu-toolchain
git submodule update --init
rm -rf ./gcc
git clone https://gcc.gnu.org/git/gcc.git
cd gcc
git branch -r
git checkout releases/gcc-15
cd ..
mkdir build && cd build
# --prefix    指定的是工具链要放在的地方,注意要有写权限
# --with-arch 指定的是拓展,默认是 rv64gc
# 无需 make install,make 后会直接放到指定的目录
# 编译可能非常非常慢
../configure --prefix=/home/atal/riscv-toolchain/ --with-arch=rv64gcv --enable-multilib
make -j4
	\end{verbatim}

为了方便我们可以把工具链的目录加到 \texttt{PATH} 变量里,这样就可以直接调用。

	\begin{verbatim}
export PATH=$PATH:/home/atal/riscv-toolchain/bin
	\end{verbatim}

	\subsection{pk 和 spike}
	\texttt{spike} 是 \texttt{riscv} 自行开发的一个模拟器
	
	首先编译 pk(proxy kernel)
	\begin{verbatim}
git clone https://github.com/riscv/riscv-pk.git
cd riscv-pk
		
mkdir build && cd build
../configure --prefix=/home/atal/riscv-toolchain/ --host=riscv64-unknown-elf --with-arch=rv64gcv
make -j4
make install
	\end{verbatim}
	
	然后是 spike 本身
	
	\begin{verbatim}
git clone https://github.com/riscv/riscv-isa-sim.git
cd riscv-isa-sim
	
mkdir build && cd build
../configure --prefix=/opt/riscv
make -j4
make install
	\end{verbatim}
	
	老版本的 \texttt{spike} 编译的时候需要加上 \texttt{--with-isa=rv64gcv} 来启用 \texttt{v} 拓展,但现在这个选项已经删掉了,现在我们只需要在运行的时候加上 \texttt{--isa=rv64gcv} 来指定运行时的架构。
	
	使用方式是这样的:
	\begin{verbatim}
riscv64-unknown-elf-gcc ./test.c -o test
spike --isa=rv64gcv pk ./test
	\end{verbatim}
	\subsection{qemu for riscv}
	同时我们可以用 \texttt{qemu} 进行模拟,这两个选其中一个用就好。
	
	\begin{verbatim}
git clone https://github.com/qemu/qemu
cd qemu
mkdir build && cd build
../configure --target-list=riscv64-softmmu,riscv64-linux-user --prefix=/home/atal/riscv-toolchain/qemu
make -j4 
make install
	\end{verbatim}
	\section{ncnn 代码浅析}
	TODO,内容有点多可能会单独开一系列文章。
	\section{Riscv v 拓展浅析}
	\subsection{Riscv v 拓展大致模式}
	和 \texttt{x86} 选用的 \texttt{SIMD} 不同,\texttt{Riscv} 使用向量指令集来进行并行优化。
	
	\texttt{v} 拓展新增了若干名称以 \texttt{v} 开头的“向量寄存器”(对 RV32V 来说,一般是 $32$ 个),这些寄存器的长度由处理器分配的向量寄存器堆大小决定,处理器会把堆均匀地划分给各个启用的向量寄存器,并且把向量寄存器能够使用的最大长度存储在寄存器 \texttt{mvl} 中。能存储的元素数进一步由存储的元素长度决定。
	
	我们可以选择性地启用或禁用部分向量寄存器,处理器会动态调整向量寄存器的长度。如假设我们有 $1024$ 字节的堆空间,并且启用全部 $32$ 个 寄存器,每个寄存器都会分配到 $32$ 字节的长度;如果我们仅启用其中两个,则每个寄存器都会变成 $512$ 字节长,并且 \texttt{mvl} 将会随之动态变化。但 \texttt{mvl} 的值只能由处理器设置,软件层面无法直接修改 \texttt{mvl} 的值。
	
	同时 \texttt{v} 拓展新增了 $7$ 个非特权 \texttt{CSR} 寄存器:\texttt{vstart},\texttt{vxsat},\texttt{vxrm},\texttt{vcsr},\texttt{vl},\texttt{vtype},\texttt{vlenb}。
	
	\texttt{vxrm} 和 \texttt{vxsat} 分别是 Vector Fixed-Point Saturation Flag vxsat 和 Vector Fixed-Point Saturation Flag,他们都是 \texttt{vcsr} 中对应位的镜像。这些暂且掠过。
	
	\texttt{vl} 设定了每次向量操作所操作的元素数量,我们只会操作从开头开始 \texttt{vl} 个元素,可以通过 \texttt{setvl} 指令设定。\texttt{vstart} 则进一步设置了向量操作会从哪一个元素开始进行操作(注意不会向后顺延),注意每次向量操作之后 \texttt{vstart} 都会被清零。
	
	\texttt{vlenb} 是以字节为单位的向量寄存器长度。
	
	\texttt{vtype} 中包含 \texttt{vill},\texttt{vma},\texttt{vta},\texttt{vsew} 和 \texttt{vlmul} 五个字段。\texttt{vsew} 代表了每个元素的长度,\texttt{vlmul} 涉及到 \texttt{v} 拓展的另一个机制,多个向量寄存器可以进行拼接组合获取更长的向量,\texttt{vlmul} 则是寄存器拼接的数量。
	
	\texttt{vma} 和 \texttt{vta} 分别指示了被 \texttt{mask} 的(之后会提到)元素和不在操作范围内的元素会被如何处理,分为“undisturbed” 和 “agnostic”。值为 $0$ 代表 “undisturbed” 会全部保持原值,而 值为 $1$ 代表 “agnostic” 既可能保持原值也可能全部写 $1$。
	\subsection{常用指令}
	\includegraphics[width=0.80\textwidth]{https://cdn.luogu.com.cn/upload/image_hosting/jh7douvh.png}
	\subsubsection{存取}
	我们使用 \texttt{vld} 从内存中读取连续的一段进入向量寄存器。如当 \texttt{vsew} 为 $32$ 时,\texttt{vld v0, 0(a0)} 会从 \texttt{a0} 指向的地址开始,读取 \texttt{a0,a0 + 4, a0 + 8...} 直至 \texttt{vl} 设置的上限。
	
	同时我们可以稀疏地读取数据,\texttt{vlds v0, 0(a0), a1} 会读取 \texttt{a0, a0 + a1, a0 + 2a1} 直至上限。或者将 $offset$ 存入另一向量寄存器,通过 \texttt{vldx v0, a0, v1} 读取。
	
	存入内存只需将 \texttt{vld} 改为 \texttt{vst}。
	\subsubsection{操作}
	操作的格式基本是 \texttt{v + 操作名 + 操作种类后缀}。操作分为向量与向量操作(\texttt{.vv} 后缀)和向量与标量(\texttt{.vs} 后缀)操作两种。
	
	\subsection{Mask}
	向量架构通过掩码(Mask)的方式来实现一些分支操作。RVV 拓展提供了 $8$ 个向量谓词寄存器 \texttt{vp\{$i$\}},我们可以将其视作一列 bool 值。我们可以在他们之间进行逻辑运算(\texttt{vp\{and, or, xor, etc.\}}),也可以将向量寄存器进行比较等操作的结果存入。我们在进行向量寄存器之间的操作时可以额外一共一个谓词寄存器,当谓词寄存器一个位置的值为 $1$ 时这个位置不会进行操作,而是会根据 \texttt{vma} 位置的值来选择保持原值还是置 $1$。
	
	\subsection{Intrinsic}
	简单来说就是给编程语言提供了一个便于调用和管理的接口,让你不用在项目里写大量的内联汇编。
	
	文档主要在 \href{https://github.com/riscv-non-isa/rvv-intrinsic-doc/}{这},另外有个\href{https://dzaima.github.io/intrinsics-viewer/}{超好用的网站}可以帮你快速搜索以及熟悉用途,向 Github 用户 dzaima 致敬。
	
	\subsection{类型系统}
	\texttt{Intrinsic} 最方便的一点就是提供了类型系统,使得我们能像操作变量一样操作向量寄存器,而不是把自己训练成寄存器管理大师。 一个寄存器的大概格式为 \texttt{v<type><vlmul>_t}。
	
	其中 \texttt{type} 可以取 \texttt{int\{8/16/32/64\} | uint\{8/16/32/64\} | float\{16/32/64\} | bool\{1/2/4/8/16/32/64\}},一看就知道什么意思!
	
	\texttt{vlmul} 就是多少个寄存器拼在一起,大于一用 \texttt{m\{1/2/4/8\}} 来表示,小于的用 \texttt{mf\{2/4/8\}} 来表示拆成 \texttt{2/4/8} 份。注意 \texttt{bool} 类型不能拼,也不需要写 \texttt{<vlmul>} 这项,举个例子你应该写 \texttt{vbool1_t}。
	
	另外你还可以在 \texttt{<vlmul>} 后面加个 \texttt{x2/4/8} 来构成元组类型,但是似乎没什么用。
	
	\subsection{指令}
	指令的基本格式 \texttt{__riscv_<instruction>_<operand>_<return_type>_<policy>}。举个例子 \texttt{vint32m1_t __riscv_vadd_vv_i32m1_m(vbool32_t vm, vint32m1_t vs2, vint32m1_t vs1,
		size_t vl);}
	
	\texttt{instruction} 就是指令名。\texttt{operand} 代表操作类型,如 \texttt{vv}。
	
	这里的 \texttt{return_type} 采用简写,即 \texttt{<i\{8, 16, 32, 64\} | u\{8, 16, 32, 64\} | f\{16, 32, 64\}><m\{1, 2, 4, 8\} | mf\{2, 4, 8\}>}。
	
	\texttt{policy} 用来表示 \texttt{vta} 和 \texttt{vma} 这两个寄存器,决定被 \texttt{mask} 和尾部的位置该怎么处理。
	\begin{itemize}
		\item 留空:不使用 mask,\texttt{vta = 1}
		\item \texttt{_tu}:不使用 mask,\texttt{vta = 0}
		\item \texttt{_m} 使用 mask,\texttt{vta = 1},\texttt{vma = 1}
		\item \texttt{_tum} 使用 mask,\texttt{vta = 0},\texttt{vma = 1}
		\item \texttt{_mu} 使用 mask,\texttt{vta = 1},\texttt{vma = 0}
		\item \texttt{_tumu} 使用 mask,\texttt{vta = 0},\texttt{vma = 0}
	\end{itemize}
	\subsection{有关 vl}
	\texttt{Intrinsic} 提供了 \texttt{__riscv_vsetvl_*},将可用的 \texttt{vl} 和你提供的要计算的长度取 $min$,确保得到的始终是合法的 vl。
	\section{实际开发、测试}
	\subsection{为 RISC-V 编译}
	我们先来讲讲怎么编译以及怎么跑测试。在 \texttt{ncnn/toolchains} 下有若干个针对不同平台进行设置的 \texttt{cmake} 文件,比如 \texttt{riscv64-unknown-linux-gnu.toolchain.cmake}。和 RISCV 有关的这几个基本只有选用的编译器不同。首先确保你设置里 \texttt{RISCV_ROOT_PATH} 这个环境变量。
	\begin{verbatim}
export RISCV_ROOT_PATH=<path_to_your_riscv_toolchain>
mkdir build && cd build
cmake -DCMAKE_TOOLCHAIN_FILE=../toolchains/riscv64-unknown-linux-gnu.toolchain.cmake -DCMAKE_BUILD_TYPE=debug -DNCNN_COVERAGE=ON -DNCNN_RUNTIME_CPU=OFF -DNCNN_RVV=ON -DNCNN_ZFH=ON -DNCNN_ZVFH=ON -DNCNN_OPENMP=ON -DNCNN_BUILD_TOOLS=OFF -DNCNN_BUILD_EXAMPLES=OFF -DNCNN_BUILD_TESTS=ON ..
// 我们这里把 CMAKE_BUILD_TYPE 设置为 debug 是为了方便调试,如果要跑 benchmark 建议切换到 release
cmake --build . -j 8
TESTS_EXECUTABLE_LOADER=qemu-riscv64 TESTS_EXECUTABLE_LOADER_ARGUMENTS="-cpu;rv64,v=true,zfh=true,zvfh=true,vlen=256,elen=64,vext_spec=v1.0;-L;<path_to_your_riscv_toolchain>/sysroot" ctest --output-on-failure -j 8
	\end{verbatim}
	\subsection{优化算子}
	如果你想添加一个全新的算子你需要先阅读 \href{https://github.com/Tencent/ncnn/wiki/add-custom-layer.zh}{这两篇}\href{https://github.com/Tencent/ncnn/wiki/how-to-implement-custom-layer-step-by-step}{文档},简而言之你要先实现一份 native 的版本并且在 \texttt{CMakeList} 里注册。
	
	\subsubsection{以防你不知道 v, zfh, zvfh 和 xtheadvector 的故事}
	\texttt{v}, \texttt{zfh}, \texttt{zvfh} 和 \texttt{xtheadvector} 是 RISC-V 的四个拓展,简单介绍一下:
	\begin{itemize}
		\item \texttt{v}:上面提到的向量拓展
		\item \texttt{zfh} 半精度浮点数拓展,支持 16bit 的浮点数
		\item \texttt{zvfh} 半精度浮点数的向量拓展
		\item \texttt{xtheadvector} 其实就是 rvv-0.7.1 指令集,虽然已经被弃用但是作为曾经常用的标准,现在还有很多设备在用这套指令,于是给它塞到了 \texttt{xthead} 里。有 \texttt{zfh} 的功能和 \texttt{v} 拓展和 \texttt{zvfh} 的大部分功能。
	\end{itemize}
	也就是说 \texttt{xtheadvector} 不支持一些现在的 \texttt{v} 拓展的 \texttt{Intrinsic}。\href{https://github.com/XUANTIE-RV/thead-extension-spec/blob/master/xtheadvector/intrinsics.adoc}{点击这里看更多}。
	
	各个厂商的各个设备都有可能随机地实现了其中几个拓展,因此我们编译的时候会把代码复制出好几份用不同的参数分别编译。
	
	我们需要分别实现单精浮点数和半精浮点数的代码(\texttt{layer_riscv.cpp} 和 \texttt{layer_riscv_zfh.cpp}),前者会编译成\texttt{rv64gc},\texttt{rv64gcv},\texttt{rv64gc_xtheadvector} 三个版本,后者会编译成 \texttt{rv64gc_zfh},\texttt{rv64gcv_zfh_zvfh},\texttt{rv64gc_zfh_xtheadvector}。
	
	你的代码需要同时支持这几种情况。我们需要通过 \texttt{__riscv_vector}、\texttt{__riscv_xtheadvector} 和 \texttt{__riscv_zvfh} 这几个宏来判断各个拓展是否开启,然后写出对应的正确代码。
	
	\subsubsection{你具体要做的}
	首先我们在 src/layer/riscv 下,先建立 \texttt{yourlayer_riscv.h}。在 ncnn namesoace 里面新建一个 \texttt{Yourlayer_riscv} 继承自 native 版本的 Yourlayer 类。然后实现你要优化的函数。
	
	例如
	\begin{verbatim}
namespace ncnn {
	
	class BNLL_riscv : public BNLL
	{
		public:
		BNLL_riscv();
		
		virtual int forward_inplace(Mat& bottom_top_blob, const Option& opt) const;
		
		protected:
		#if NCNN_ZFH
		int forward_inplace_fp16s(Mat& bottom_top_blob, const Option& opt) const;
		#endif
	};
	
} // namespace ncnn	
	\end{verbatim}
	
	然后在 \texttt{yourlayer_riscv.cpp} 和 \texttt{yourlayer_riscv_zfh.cpp} 里分别实现 fp32 和 fp16 的版本:
	
\begin{verbatim}

namespace ncnn {
	
	BNLL_riscv::BNLL_riscv()
	{
		#if __riscv_vector
		support_packing = true;
		#endif // __riscv_vector
		#if NCNN_ZFH
		#if __riscv_vector
		support_fp16_storage = cpu_support_riscv_zvfh();
		#else
		support_fp16_storage = cpu_support_riscv_zfh();
		#endif
		#endif
	}
	
	int BNLL_riscv::forward_inplace(Mat& bottom_top_blob, const Option& opt) const
	{
		#if NCNN_ZFH
		int elembits = bottom_top_blob.elembits();
		
		if (opt.use_fp16_storage && elembits == 16)
		{
			return forward_inplace_fp16s(bottom_top_blob, opt);
		}
		#endif
		
		int w = bottom_top_blob.w;
		int h = bottom_top_blob.h;
		int d = bottom_top_blob.d;
		int channels = bottom_top_blob.c;
		int elempack = bottom_top_blob.elempack;
		int size = w * h * d * elempack;
		
		#pragma omp parallel for num_threads(opt.num_threads)
		for (int q = 0; q < channels; q++)
		{
			float* ptr = bottom_top_blob.channel(q);
			
			#if __riscv_vector // 判断是否启用了 v 拓展
			int n = size;
			while (n > 0)
			{
				size_t vl = __riscv_vsetvl_e32m8(n);
				
				vfloat32m8_t _p = __riscv_vle32_v_f32m8(ptr, vl);
				vbool4_t _mask = __riscv_vmfgt_vf_f32m8_b4(_p, 0.f, vl);
				
				#if __riscv_xtheadvector // 判断是否为 xtheadvector
				vfloat32m8_t _comm = __riscv_vfsgnjx_vv_f32m8(_p, _p, vl);
				_comm = __riscv_vfsgnjn_vv_f32m8(_comm, _comm, vl);
				#else
				vfloat32m8_t _comm = __riscv_vfsgnjn_vv_f32m8_mu(_mask, _p, _p, _p, vl);  // 这条指令在 xtheadvector 里不存在
				#endif
				_comm = exp_ps(_comm, vl);
				_comm = __riscv_vfadd_vf_f32m8(_comm, 1.f, vl);
				_comm = log_ps(_comm, vl);
				
				#if __riscv_xtheadvector // 判断是否为 xtheadvector
				vfloat32m8_t _res = __riscv_vfadd_vv_f32m8(_comm, _p, vl);
				_res = __riscv_vmerge_vvm_f32m8(_comm, _res, _mask, vl);
				#else
				vfloat32m8_t _res = __riscv_vfadd_vv_f32m8_mu(_mask, _comm, _comm, _p, vl); // 这条指令在 xtheadvector 里不存在
				#endif
				__riscv_vse32_v_f32m8(ptr, _res, vl);
				
				ptr += vl;
				n -= vl;
			}
			#else  // __riscv_vector 
			// 标量版本
			for (int i = 0; i < size; i++)
			{
				if (*ptr > 0)
				*ptr = *ptr + logf(1.f + expf(-*ptr));
				else
				*ptr = logf(1.f + expf(*ptr));
				++ptr;
			}
			#endif // __riscv_vector
		}
		
		return 0;
	}
}
\end{verbatim}
	
	\subsection{玄铁你真的是}
	然后我们需要测试这些代码是否在玄铁的各个设备上能够正常运行。为什么?因为用它的板子太多了。而且 ncnn ci 也测了(什么
	
	首先我们要再下一份玄铁家的 toolchain 和 qeum。你可以从\href{https://www.xrvm.cn/community/download}{这里}找到形如 \texttt{Xuantie-900-gcc-linux-6.6.0-glibc-x86_64-V3.1.0-20250522.tar.gz} 的东西和 \texttt{Xuantie-qemu-x86_64-Ubuntu-20.04-V5.2.6-B20250415-1115.tar.gz} 这种东西。你看到这里的时候可能会有更新的版本。
	
	不知道为什么玄铁的工具链还要手机登录才能下载。好抽象。不懂哦。
	
	然后你会在 ncnn/toolchains 下找到 \texttt{\{c906, c908, c910\}-v310.toolchain.cmake} 这种文件(你看到这里的时候可能会有更新的版本。)我们像刚刚一样进行编译和测试,但是这次要把 \texttt{riscv-root-path} 改成玄铁的路径。
	\begin{verbatim}
export RISCV_ROOT_PATH=<your_xuantie_toolchain_path>
mkdir build && cd build
  cmake -DCMAKE_TOOLCHAIN_FILE=../toolchains/c906-v301.toolchain.cmake -DCMAKE_BUILD_TYPE=release \
	-DNCNN_OPENMP=OFF -DNCNN_THREADS=OFF \
	-DNCNN_RUNTIME_CPU=OFF \
	-DNCNN_RVV=OFF \
	-DNCNN_XTHEADVECTOR=ON \
	-DNCNN_ZFH=ON \
	-DNCNN_ZVFH=OFF \
	-DNCNN_SIMPLEOCV=ON -DNCNN_BUILD_EXAMPLES=ON -DNCNN_BUILD_TESTS=ON ..
cmake --build . -j 8
	TESTS_EXECUTABLE_LOADER=<path_to_your_xuantie_qemu> TESTS_EXECUTABLE_LOADER_ARGUMENTS="-cpu;c906fdv" ctest --output-on-failure -j 8
\end{verbatim}

	然后玄铁有一堆奇奇怪怪的上游 bug。\href{https://github.com/XUANTIE-RV/xuantie-gnu-toolchain/issues}{点击即看}。包括但不限于尾部不打扰不生效啊,fp16 和 fp32 互转会段错误啊。被创了很多次,无力吐槽了。
\end{document}
