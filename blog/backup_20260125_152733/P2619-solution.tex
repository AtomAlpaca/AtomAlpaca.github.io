\documentclass{article}
\usepackage[UTF8]{ctex}
\usepackage{hyperref}
\usepackage{fancyvrb}
\usepackage{amsmath}
\hypersetup {
	colorlinks = true,
	linkcolor  = black,
}
\newcommand{\[}{\begin{align*}}
\newcommand{\]}{\end{align*}}
	
\author {AtomAlpaca}
\title {「题解」 P2619 [国家集训队] Tree I}

\begin{document}
	\tableofcontents
\subsection{题目}
给出一个无向带权连通图,每条边是黑色或白色。求一棵最小权的恰好有 $need$ 条白色边的生成树。

\href{https://www.luogu.com.cn/problem/P2619}{link}

\subsection{题解}

凸性的证明思路和最小度限制生成树类似。首先对于每个白边 $(u, v, w)$,考虑新增一条黑边 $(u, v, +\infty)$,防止白边过少时无法得到生成树。考虑当前有 $k - 1$ 条白边的生成树,现在要强行加进去一个白边,设其权值为 $w_1$,这使得原来的生成树上多出来一个环,我们选这个环上所有黑边中权值最大的一个删掉,设其权值为 $w_2$。我们每次要选择加入后构成的环中有黑边,且权值增量 $w = w_1 - w_2$ 最小的一个白边。不难发现这两个限制随着白边的加入都是越来越紧的——第 $k$ 次未加入生成树的白边第 $k + 1$ 次可能加入了,第 $k$ 次存在的黑边第 $k + 1$ 次可能不存在。

因此如果第 $k + 1$ 次加入时的增量小于 $k$ 次,那一定可以交换 $k, k + 1$ 次的选边方案使得第 $k$ 次更优,因此加入白边后的权值增量不增,故该问题有凸性,可以用 wqs 二分解决。

于是二分一个增量,每次将所有白边权值加上这个增量,再和黑边一起跑最小生成树,根据最小生成树中白边的数量调整增量即可。

2025/11/20 补:经 \href{https://www.luogu.com/user/223298/}{@do\_while\_true} 老师提醒补充一下增量构造的正确性,也即我们总能通过删除一条黑边加入一条白边得到从 $f(k - 1)$ 得到 $f(k)$ 的答案

设图拟阵 $\mathcal M = {\left \langle E, \mathcal I \right \rangle}$,$T_{k - 1}$ 是 $k - 1$ 时的一个最优生成树,$T_{k}$ 同理。二者都是 $\mathcal M$ 的一个基,由强基交换定理我们总有一个完美匹配 $\{(e, f)\}$ 其中 $e_i \in T_{k - 1}, f_i \in T_{k}$ 满足 $\forall i, T_{k - 1} \cup \{f_i\} \backslash \{e_i\}$ 和 $T_{k} \cup \{e_i\} \backslash \{f_i\}$ 也是 $\mathcal M$ 的基。其中一定存在一对边 $(f, g)$ 满足 $f$ 是白边,$e$ 是黑边,交换这对边也即删除一条黑边加入一条白边。

设
\[
	T_{\alpha} &= T_{k - 1} \cup \{ f \} \backslash \{ e \} \\
	T_{\beta} &= T_{k} \cup \{ e \} \backslash \{ f \}
\]
则 
\[
	w(T_{\alpha}) &= w(T_{k - 1}) + w(f) - w(e)
	\\
	&=f(k - 1) + w(f) - w(e)
\]

同理有 $w(T_{\beta}) = f(k) - (w(f) - w(e))$

由于 $w(T_{\alpha}) \ge f(k), w(T_{\beta}) \ge f(k - 1)$,带入

\[
	f(k) - f(k - 1) &\le w(f) - w(e)
	\\
	f(k) - f(k - 1) &\le -(w(f) - w(e))
\]
取等,故仅交换这一组边能达到最优,证毕

上高三好累,我想睡觉。

\subsection{代码}

\begin{Verbatim}
#include <bits/stdc++.h>

using std::cin;
using std::cout;

const int MAX = 1e5 + 5;

int n, m, k, u, v, w, c, cnt, res, ans;
int f[MAX];
struct E { int u, v, w, c; } e[MAX];

int find(int x) { if (f[x] == x) { return x; } return f[x] = find(f[x]); }
bool cmp(E a, E b) { if (a.w == b.w) { return a.c < b.c; } return a.w < b.w; }

void check(int x)
{
	res = cnt = 0;
	for (int i = 1; i <= m; ++i) { if (e[i].c == 0) { e[i].w += x; } }
	for (int i = 1; i <= n; ++i) { f[i] = i; }
	std::sort(e + 1, e + m + 1, cmp);
	for (int i = 1, tmp = 0; i <= m and tmp != n - 1; ++i)
	{
		int fu = find(e[i].u), fv = find(e[i].v);
		if (fu == fv) { continue; }
		f[fv] = fu; if (e[i].c == 0) { ++cnt; } res += e[i].w; ++tmp;
	}
	for (int i = 1; i <= m; ++i) { if (e[i].c == 0) { e[i].w -= x; } }
}

int main()
{
	cin >> n >> m >> k;
	for (int i = 1; i <= m; ++i)
	{
		cin >> u >> v >> w >> c;
		e[i] = {u + 1, v + 1, w, c};
	}
	int l = -111, r = 111, mid = 0;
	while (l <= r)
	{
		mid = (l + r) >> 1;
		check(mid);
		if (cnt < k) { r = mid - 1; } else { l = mid + 1; ans = res - k * mid; }
	}
	cout << ans;
	return 0;
}
\end{Verbatim}

\end{document}