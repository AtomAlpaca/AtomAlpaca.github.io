\documentclass{article}
\usepackage[UTF8]{ctex}
\usepackage{hyperref}

\author {AtomAlpaca}
\title  {2021}

\begin{abstract}
	Mi pretendis vin ami, pretendis vin malami
	senhelpe, kiel mi, mia koro enhavis nenion
	Mi vivas, tiam mi mortos en la paŝo de la tempo
	Por Kiun mi transdonos tiun poemon
	Kiu kantos al ĉi tiu loko
\end{abstract}

\begin{document}
	\tableofcontents

	\subsection{五月}
	年度总结其实早就想写了。
	\par
	我想了很多年度总结要怎么写,但是打开文本编辑器,看着灰暗的背景,却又写不出什么了。

	\par
	也许是生活实在枯槁吧,也许是不敢面对过去吧。

	\par
	冬雪煎成雨,秋叶落成花。等到我准备好整理我的 2021 时,已经五月了。

	\subsection{我不知道}
	秋天的时候确诊了抑郁和焦虑。

	\par
	为什么呢?

	\par
	是因为孤单吗?是因为陡增的学习压力吗?是因为紧张的家庭关系吗?是因为对自我的憎恶吗?

	\par
	我不知道。

	\par
	我数不清这一年我说了多少个“我不知道”了。我拿它回答父母的询问,搪塞老师同学的关心,填写心理咨询师的问卷。

	\par
	也用来回答我的内心。

	\par
	我怎么了?

	\par
	我不知道。

	\begin{quote}
		「她会有自己的想法,会有自己的志向,会不断动摇,从来没有清晰过……但,某一天她会发现,自己想要这么做,没有为什么,也一定会这么做。 \newline
    	于是她永远美丽下去,她的内心是一团巨大的灼烧着她自己的火焰。」
	\end{quote}

	\subsection{路}
	最终还是走上了算法竞赛这条路。
	\par
	我不知道这是否正确的选择。2020 年的末尾,我还在学习 Web 相关的东西,运营自己的论坛,学 Java 和前端之类;2021 年的开头,我却将自己积累下的一切弃著角落。

	\par
	但是我发现……也许算法这种底层,抽象,繁复却浪漫的东西更能让我开心。

	\par
	我总是对古老或独特甚至过时,但仍然行之有效的东西怀有奇怪的好感。

	\par
	结果……还算说得过去吧, CSP-J 拿了一等,S 组因为最后一版代码没保存上导致 100 -> 0,爆零了。

	\par
	我不知道我走竞赛这条路能走多远,在这样一个弱省,在这样一个弱校,竞赛似乎没有出路,也没什么人会帮我。

	\par
	但是我还想走下去,无论是为了逃避自己鄙夷的应试教育也好,无论是出于对算法的兴趣也好。

	\par
	我很明白,我最大的缺点不是找不到正确的路,而是没有沿着正确的路一直走下去的勇气和魄力,差几步就解出来的数学压轴题也好,构思了好久却没写出来的博文也好,写了删删了又写的小项目也好。

	\par
	走下去吧,哪怕这不是最好的道路。

	\par
	「Whatever happens, I’ll leave it all to chance.」

	\subsection{疏渺}
	「听哪,死一般沉寂!——正是高峰时刻! \\
	我醒着入睡了, \\
	从不堪忍受的梦境逃到了温柔的现实。」 \\
	————汉德克《颠倒的世界》

	\par
	「疏渺」是我给自己找的名字。

	\par
	这不是一个好名字。疏远,飘渺,都是悲伤凄冷的意象。寓意不好,太孤独了。

	\par
	但……我似乎就是这样的,很久了吧?现在的朋友,似乎都是 2021 年才熟络起来的。

	\par
	我曾经固执地认为,孤独是我的空气,是我生活的底色。但后来我才发现,孤独不过是童年时代一次次迁徙,将我学生时代拆的七零八落,留给我的副产物。

	\par
	我想要朋友,渴望那种有人陪在我身边的温暖,像是一池水里生长的水草摇摇晃晃,仿佛有朵莲花在我心上温柔开放,在雪下捧我一抹蔻梢绿,对我说一切都会好起来。远处,一只鸟飞向原野,我可以肆无忌惮地哭了。
	
	\par
	但是接下来的路似乎只有我自己走了。

	\par
	「缺月挂疏桐,漏断人初静。谁见幽人独往来,飘渺孤鸿影。」

	\subsection{明天}
	\par
	现在做新年展望好像太晚了。
	
	\par
	明天我会去干什么?以后我会成为什么样的人?我不知道。

	\par
	但我希望,我能学习自己喜欢的东西。

	\par
	我想学算法,学 Rust, 学 Haskell, 学数学,学编译原理……

	\par
	我希望有一天,我能行止如我自己。
\end{document}
