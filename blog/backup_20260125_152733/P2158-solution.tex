\documentclass{article}
\usepackage[UTF8]{ctex}
\usepackage{hyperref}

\author {AtomAlpaca}
\title  {「题解」 P8474 立春}
\begin{document}

\tableofcontents

\par
不知道为什么题解区都是用欧拉函数做的,个人感觉完全用不到啊。

\par
考虑将观测点作为原点,向右为 $x$ 轴 正方向,向上为 $y$ 轴正方向,那么所求即为 $2 +\sum_{i = 1}^{n-1}{\sum_{j = 1}^{n - 1} [ \gcd(i, j)=1 ]}$。

\par
考虑如何求最大公倍数为 $k$ 的数对数。考虑容斥,即为约数中含 $k$ 的数对数量,减去约数中含 $k$ 的倍数的数对数量。我们设最大公倍数为 $k$ 的数对为 $f(k)$,则有:
$$f(x) = (\lfloor \frac{n}{x}\rfloor)^2 - \sum_{i = 2}^{ix \le n} { f(ix) }$$

\par
答案即为 $f(1) + 2$,代码非常好写。
\subsection{代码}

\begin{verbatim}
#include <bits/stdc++.h>

using std::cin;
using std::cout;

typedef long long ll;
const ll MAX = 4e4 + 5;
ll n, f[MAX];

int main()
{
	cin >> n; --n; if (!n) { cout << 0; return 0; }
	for (ll i = n; i >= 1; --i)
	{
		f[i] = (n / i) * (n / i);
		for (ll j = 2; j * i <= n; ++j) { f[i] -= f[i * j]; }
	}
	cout << f[1] + 2;
	return 0;
}
\end{verbatim}
\end{document}
