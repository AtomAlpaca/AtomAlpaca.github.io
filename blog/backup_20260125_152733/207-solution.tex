\documentclass{article}
\usepackage[UTF8]{ctex}
\usepackage{hyperref}
\usepackage{graphicx}

\author {AtomAlpaca}
\title  {「题解」UOJ 207. 共价大爷游长沙} 
\begin{document}
\subsection{题意}
\href{https://uoj.ac/problem/207}{link}

\subsection{题解}

如果一条路径 $u, v$ 在所有路径上,那么我们把 $u$ 当作根拎起来,$v$ 的子树中一定包括且仅包括路径的其中一个顶点。

我们发现这个东西很像异或,于是考虑星战那题的 Trick, 我们给每个路径随机分配一个权值,然后异或到两个顶点上。如果把 $u$ 当作根拎起来之后$v$ 的子树点集异或和等于所有边的异或和,那么我们可以认为它包括且仅包括了所有路径的一个顶点。考虑到加删边,考虑用 LCT 维护这个东西。

然后 LCT 维护子树信息要用点 Trick, 具体来说就是对每个结点都多维护一个虚边连接的“虚子树”的信息,然后在 pushup 和 access、link 这些会改变轻重链信息的操作中更新一下就行了。具体可以看代码。

\begin{verbatim}
void pu(int x) { t[x].v = t[t[x].c[0]].v ^ t[t[x].c[1]].v ^ t[x].p; }

void acc(int x)
{
	int lst = 0;
	while (x)
	{
		splay(x); t[x].p ^= t[lst].v ^ t[t[x].c[1]].v; t[x].c[1] = lst;
		pu(x); lst = x; x = t[x].f;
	}
}

void link(int x, int y) { mkrt(x); acc(y); splay(y); t[x].f = y; t[y].p ^= t[x].v; }
\end{verbatim}

\subsection{代码}

\begin{verbatim}
#include <bits/stdc++.h> 

const int MAX = 1e5 + 5;
const int MAXX = 3e5 + 5;
std::mt19937_64 rnd(time(0));

int u, v, id, n, m, x, y, typ, tot, ans;
int a[MAXX], b[MAXX], p[MAXX];

struct N { int v, p, f, c[2]; bool r; } t[MAX];

void pu(int x) { t[x].v = t[t[x].c[0]].v ^ t[t[x].c[1]].v ^ t[x].p; }
void rv(int x) { std::swap(t[x].c[0], t[x].c[1]); t[x].r ^= 1; }
void pd(int x) { if (t[x].r) { rv(t[x].c[0]); rv(t[x].c[1]); t[x].r = 0; } }
bool fx(int x) { return t[t[x].f].c[1] == x; }
bool isrt(int x) { return (!t[x].f) or (t[t[x].f].c[0] != x and t[t[x].f].c[1] != x); }
void upd(int x) { if (!isrt(x)) { upd(t[x].f); } pd(x); }

void rot(int x)
{
	int y = t[x].f, z = t[y].f, p = fx(x);
	if (!isrt(y)) { t[z].c[fx(y)] = x; }
	t[y].c[p] = t[x].c[p ^ 1];
	if (t[x].c[p ^ 1]) { t[t[x].c[p ^ 1]].f = y; }
	t[x].c[p ^ 1] = y; t[y].f = x; t[x].f = z;
	pu(y); pu(x);
}

void splay(int x)
{
	upd(x);
	while (!isrt(x))
	{
		int y = t[x].f;
		if (!isrt(y)) { if (fx(x) == fx(y)) { rot(y); } else { rot(x); } } rot(x);
	}
	pu(x);
}

void acc(int x)
{
	int lst = 0;
	while (x)
	{
		splay(x); t[x].p ^= t[lst].v ^ t[t[x].c[1]].v; t[x].c[1] = lst;
		pu(x); lst = x; x = t[x].f;
	}
}

void mkrt(int x) { acc(x); splay(x); rv(x); }
void split(int x, int y) { mkrt(x); acc(y); splay(y); }
void link(int x, int y) { mkrt(x); acc(y); splay(y); t[x].f = y; t[y].p ^= t[x].v; }
void cut(int x, int y) { mkrt(x); acc(y); splay(y); t[y].c[0] = t[x].f = 0; pu(y); }

int main()
{
	scanf("%d%d%d", &id, &n, &m);
	for (int i = 1; i <  n; ++i) { scanf("%d%d", &u, &v); link(u, v); }
	for (int i = 1; i <= m; ++i)
	{
		scanf("%d", &typ);
		if (typ == 1)
		{
			scanf("%d%d%d%d", &x, &y, &u, &v);
			cut(x, y); link(u, v);
		}
		else if (typ == 2)
		{
			scanf("%d%d", &u, &v);
			p[++tot] = rnd(); a[tot] = u; b[tot] = v; ans ^= p[tot];
			mkrt(u); t[u].p ^= p[tot]; t[u].v ^= p[tot];
			mkrt(v); t[v].p ^= p[tot]; t[v].v ^= p[tot];
		}
		else if (typ == 3)
		{
			scanf("%d", &x); int u = a[x], v = b[x]; ans ^= p[x];
			mkrt(u); t[u].p ^= p[x]; t[u].v ^= p[x];
			mkrt(v); t[v].p ^= p[x]; t[v].v ^= p[x];
		}
		else
		{
			scanf("%d%d", &u, &v);
			split(u, v); printf(t[v].p == ans ? "YES\n" : "NO\n");
		}
	}
}
\end{verbatim}
\end{document}