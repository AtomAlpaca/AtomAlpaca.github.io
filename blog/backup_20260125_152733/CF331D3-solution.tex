\documentclass{article}
\usepackage[UTF8]{ctex}
\usepackage{hyperref}
\usepackage{fancyvrb}
\usepackage{amsmath}
\hypersetup {
	colorlinks = true,
	linkcolor  = black,
}

\newcommand{\[}{\begin{align*}}
\newcommand{\]}{\end{align*}}
\renewcommand{\par}[1]{\paragraph{#1}}


\author {AtomAlpaca}
\title {「题解」CF331D3 Escaping on Beaveractor}

\begin{document}

\subsection{题意}\label{ux9898ux610f}

\href{https://codeforces.com/contest/331/problem/D3}{link}

题意简述:在坐标系中给定边界和若干有向线段,每次询问从一个点沿某个方向走
\(t\)
时间,碰到有向线段就把前进方向改为这个有向线段的方向,时间耗尽或者碰到边界就停,最终会停在哪个位置。

\subsection{题解}\label{ux9898ux89e3}

首先一个朴素的想法是,维护每个格子开始往左/右/上/下跳 \(2^k\)
步,会跳到那个点上,然后倍增,维护的时候暴力跳,然后查这个点是否在某个有向线段上。

然后我们发现跳到两条线段的交叉点没法判断往哪边走,于是还要维护每个状态是从哪个方向转移来的;然后我们又发现这个办法的复杂度是
\(O(b^2 \log t \log n)\) 的,只能通过弱化版,于是考虑优化。

然后我们发现,对于每个点维护是不必要的,因为一条有向线段上的所有点的轨迹是相同的,于是我们考虑转而维护线段。

不妨把每个询问点看作一个长度为 \(0\) 的有向线段。维护 \(f_{x, y}\)
表示从线段 \(x\) 开始,走了除自己之外 \(2^y\)
条线段,最后到达了哪个线段;\(g_{x,y}\) 表示从线段 \(x\) 开始,走完了
\(f_{x, y}\) 条线段的路程。

对于 \(y \ge 1\) 的情况,我们可以很容易地倍增求出;对于 \(y = 0\)
的情况,我们实际上是要求沿着每条线段走,第一个碰到的线段是哪一个。

我们把四个方向上的线段分别求,以向上的线段为例:我们先将所有线段按照两个端点中较小的纵坐标升序排序,依次将每个线段纳入考虑,设这个线段的横坐标范围是
\(l\sim r\),则已经纳入考虑的线段中,所有横坐标在此范围内,方向为上,且还没确定答案的线段,碰到的第一个线段一定是当前的这个线段。其余的方向也是类似的求法。

于是我们对每个询问的``线段''不断跳倍增数组,直至没法继续跳。这时还有两种情况:

\begin{enumerate}
\def\labelenumi{\arabic{enumi}.}
\tightlist
\item
  沿着最后的线段方向一直走,直到时间耗尽;
\item
  沿着最后的线段方向一直走,走到另一条线段上,但是剩余时间不够走到这条线段的末尾,所以倍增时没有跳到这条线段上。
\end{enumerate}

对于第二种情况,我们判断出来之后``透支''时间,强制跳到另一条线段上,然后再沿着反方向把时间``补回来''即可。

时间复杂度 \(O((n + q)\log t)\),代码写起来略显繁琐。

\subsection{代码}\label{ux4ee3ux7801}

\begin{verbatim}
#include <bits/stdc++.h>
 
typedef long long ll;
const int MAX = 2e5 + 5;
const int MAXX = 51;
const ll INF = 1e15 + 5;
using std::min;
using std::max;
typedef std::multimap <ll, ll> MP;
typedef MP::iterator IT;
 
struct N
{
  ll sx, sy, tx, ty, dx = 0, dy = 0;
} a[MAX];
 
bool cmpx0(ll x, ll y) { return min(a[x].sx, a[x].tx) == min(a[y].sx, a[y].tx) ? x > y : min(a[x].sx, a[x].tx) > min(a[y].sx, a[y].tx); }
bool cmpx1(ll x, ll y) { return max(a[x].sx, a[x].tx) == max(a[y].sx, a[y].tx) ? x > y : max(a[x].sx, a[x].tx) < max(a[y].sx, a[y].tx); }
bool cmpy0(ll x, ll y) { return min(a[x].sy, a[x].ty) == min(a[y].sy, a[y].ty) ? x > y : min(a[x].sy, a[x].ty) > min(a[y].sy, a[y].ty); }
bool cmpy1(ll x, ll y) { return max(a[x].sy, a[x].ty) == max(a[y].sy, a[y].ty) ? x > y : max(a[x].sy, a[x].ty) < max(a[y].sy, a[y].ty); }
 
ll n, b, q;
char op;
ll tm[MAX], h[MAX];
ll f[MAX][MAXX], g[MAX][MAXX];
 
void solve(ll dx, ll dy, bool (*cmp)(ll, ll))
{
  for (int i = 1; i <= n + q; ++i) { h[i] = i; }
  std::sort(h + 1, h + n + q + 1, cmp);
  MP mp; mp.clear();
  for (int i = 1; i <= n + q; ++i)
  {
    if (h[i] <= n)
    {
      ll t1, t2;
      if (dx != 0) { t1 = a[h[i]].sy, t2 = a[h[i]].ty; } else { t1 = a[h[i]].sx, t2 = a[h[i]].tx; }
      if (t1 > t2) { std::swap(t1, t2); }
      IT i1 = mp.lower_bound(t1), i2 = mp.upper_bound(t2);
      for (IT j = i1; j != i2; mp.erase(j++))
      {  
        ll p = j -> second;
        f[p][0] = h[i]; g[p][0] = abs(a[p].tx - a[h[i]].tx) + abs(a[p].ty - a[h[i]].ty);
      }
    }
    if (a[h[i]].dx == dx and a[h[i]].dy == dy)
    {
      mp.insert({dx != 0 ? a[h[i]].ty : a[h[i]].tx, h[i]});
    }
  }
}
 
int main()
{
  scanf("%lld%lld", &n, &b);
  for (int i = 1; i <= n; ++i)
  {
    scanf("%lld%lld%lld%lld", &a[i].sx, &a[i].sy, &a[i].tx, &a[i].ty); 
    a[i].dx = a[i].tx - a[i].sx == 0 ? 0 : a[i].tx - a[i].sx > 0 ? 1 : -1; 
    a[i].dy = a[i].ty - a[i].sy == 0 ? 0 : a[i].ty - a[i].sy > 0 ? 1 : -1;
  }
  scanf("%lld", &q);
  for (int i = n + 1; i <= n + q; ++i)
  {
    scanf("%lld%lld %c %lld", &a[i].sx, &a[i].sy, &op, &tm[i]); a[i].tx = a[i].sx; a[i].ty = a[i].sy;
    if (op == 'L') { --a[i].dx; } if (op == 'R') { ++a[i].dx; } if (op == 'U') { ++a[i].dy; } if (op == 'D') { --a[i].dy; }
  }
  solve(-1, 0, cmpx0); solve(1, 0, cmpx1); solve(0, -1, cmpy0); solve(0, 1, cmpy1);
  for (int i = 1; i <= MAXX - 1; ++i)
  {
    for (int j = 1; j <= n + q; ++j)
    {
      f[j][i] = f[f[j][i - 1]][i - 1];
      g[j][i] = std::min(INF, g[j][i - 1] + g[f[j][i - 1]][i - 1]);
    }
  }
  for (int i = n + 1; i <= n + q; ++i)
  {
    int p = i;
    for (int j = MAXX - 1; j >= 0; --j)
    {
      if (!f[p][j]) { continue; }
      if (g[p][j] <= tm[i]) { tm[i] -= g[p][j]; p = f[p][j]; }
    }
    if (f[p][0] and tm[i] >= a[p].dx * (a[f[p][0]].tx - a[p].tx) + a[p].dy * (a[f[p][0]].ty - a[p].ty))
    {
      tm[i] -= g[p][0]; p = f[p][0];
    }
    printf("%lld %lld\n", std::min(std::max(a[p].tx + a[p].dx * tm[i], 0ll), b),
                          std::min(std::max(a[p].ty + a[p].dy * tm[i], 0ll), b));
  }
  return 0;
}
\end{verbatim}

\end{document}